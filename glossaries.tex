\makeglossaries

% ----------------------------------------------------------
% Define a glossary type for symbols
\newglossary[slg]{symbolslist}{syi}{syg}{List of Symbols}

% ----------------------------------------------------------
% Define abbreviations (default glossary type)
\newacronym{3d}{3D}{Three Dimensional}
\newacronym{fem}{FEM}{Finite Element Method}

% ----------------------------------------------------------
% Define symbols (in our new symbolslist glossary)
\newglossaryentry{Cg}{
    type=symbolslist,
    name={$C_g$},
    description={contiguity}
}
\newglossaryentry{phid}{
    type=symbolslist,
    name={$\phi_d$},
    description={dihedral angle}
}
\newglossaryentry{Ulr}{
    type=symbolslist,
    name={$U^{LR}$},
    description={Ewald Energy for Reciprocal (Long-Range) Space}
}
\newglossaryentry{Usr}{
    type=symbolslist,
    name={$U^{SR}$},
    description={Ewald Energy for Real (Short-Range) Space}
}
\newglossaryentry{Us}{
    type=symbolslist,
    name={$U^{S}$},
    description={Self Interaction Correction to Ewald Energy}
}

\printglossary[type=\acronymtype, title=List of Abbreviations]

% Print the symbols
\printglossary[type=symbolslist, title=List of Symbols]
This thesis uses the \gls{fem} approach in \gls{3d} simulations.  
Furthermore, the parameter \gls{Cg} denotes contiguity, which influences the dihedral angle \gls{phid}.  
The energies calculated, such as \gls{Ulr}, \gls{Usr}, and \gls{Us}, are essential in determining the stability of the system.
