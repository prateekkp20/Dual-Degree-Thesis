\documentclass[11pt, a4paper, oneside]{Thesis} % Paper size, default font size and one-sided paper
% \usepackage{floatrow}
% \floatsetup[table]{capposition=top}
\usepackage{algorithm2e}
\usepackage{amsmath}
\usepackage{array}
\usepackage[english]{babel}
\usepackage{blindtext}
\usepackage{caption,subcaption}
\usepackage{csquotes}
\usepackage{float}
\usepackage{gensymb}
\usepackage{graphicx}
\usepackage{listings,xcolor}
\usepackage{lipsum}
\usepackage{lscape}
\usepackage{multicol}
\usepackage{notoccite}
\usepackage{pdfpages}
\usepackage{ragged2e}
\usepackage{romannum}
\usepackage{rotating}
\usepackage{tabularx}
\usepackage{textcomp}
\usepackage[most]{tcolorbox}
\usepackage{upgreek}
\usepackage{wrapfig}
\RestyleAlgo{ruled}
\usepackage{xcolor}
\definecolor{codegreen}{rgb}{0,0.6,0}
\definecolor{codegray}{rgb}{0.5,0.5,0.5}
\definecolor{codepurple}{rgb}{0.58,0,0.82}
\definecolor{backcolour}{rgb}{0.95,0.95,0.92}
\lstset{
    language=C,
    backgroundcolor=\color{backcolour},   
    commentstyle=\color{codegreen},
    keywordstyle=\color{magenta},
    numberstyle=\tiny\color{codegray},
    stringstyle=\color{codepurple},
    basicstyle=\ttfamily\footnotesize,
    % keywordstyle=\color{blue},
    % commentstyle=\color{gray},
    % stringstyle=\color{red},
    numbers=left,
    numberstyle=\tiny,
    stepnumber=1,
    numbersep=5pt,
    showstringspaces=false,
    breaklines=true,
    frame=single,
    captionpos=b
}
\renewcommand{\chapterautorefname}{Chapter} % Forces capital "Chapter"
%\usepackage[open]{bookmark}

% acronyms
\usepackage{acronym}
\usepackage{array}
\newcolumntype{L}{>{\centering\arraybackslash}m{3cm}}

\newcommand{\swb}[1]{\noindent \textcolor{blue}{{\bf SB: }{\em #1}} }

%\usepackage[acronym]{glossaries}
% prints author names as small caps


\makeatletter
\AtBeginDocument{%
  \renewcommand*{\AC@hyperlink}[2]{%
    \begingroup
      \hypersetup{hidelinks}%
      \hyperlink{#1}{#2}%
    \endgroup
  }%
}
\makeatother





%\usepackage{subcaption} %incompatible with subfig
\graphicspath{{Pictures/}} % Specifies the directory where pictures are stored
\usepackage[square, numbers]{natbib} % Use the natbib reference package - read up on this to edit the reference style; if you want text (e.g. Smith et al., 2012) for the in-text references (instead of numbers), remove 'numbers' v

\hypersetup{urlcolor=black, colorlinks=true} % Colors hyperlinks in blue - change to black if annoyingv`
\title{\ttitle} % Defines the thesis title - don't touch this

\begin{document}
\makeatletter
\renewcommand*{\NAT@nmfmt}[1]{\textsc{#1}}
\makeatother

% prints author names as small caps


\frontmatter % Use roman page numbering style (i, ii, iii, iv...) for the pre-content pages

\setstretch{1.6} % Line spacing of 1.6 (double line spacing)

% Define the page headers using the FancyHdr package and set up for one-sided printing
\fancyhead{} % Clears all page headers and footers
\rhead{\thepage} % Sets the right side header to show the page number
\lhead{} % Clears the left side page header

\pagestyle{fancy} % Finally, use the "fancy" page style to implement the FancyHdr headers

\newcommand{\HRule}{\rule{\linewidth}{0.5mm}} % New command to make the lines in the title page

% PDF metadata
\hypersetup{pdftitle={\ttitle}}
\hypersetup{pdfsubject=\subjectname}
\hypersetup{pdfauthor=\authornames}
\hypersetup{pdfkeywords=\keywordnames}

%----------------------------------------------------------------------------------------
%	TITLE PAGE
%----------------------------------------------------------------------------------------

\begin{titlepage}
\begin{center}

\HRule \\[0.4cm] % Horizontal line
{\huge \bfseries \ttitle}\\[0.4cm] % Thesis title
\HRule \\[1.5cm] % Horizontal line
 
\large \textit{A thesis submitted in partial fulfilment of the \\requirements for the degree of}\\[0.3cm]
\textbf{\degreename}\\[0.3cm] % University requirement text
\textit{by}\\[0.3cm]
\textbf{\authornames} \\[0.3cm]
\textbf{(208070710)}

\vfill
\graphicspath{ {./Figures/} }
\begin{figure}[hb]
  \centering
  \includegraphics[width=0.35\linewidth]{images/iitk_logo.png}
\end{figure}

\DEPTNAME\\ % Research group name and department name
\textsc{ \UNIVNAME}\\[1.5cm] % University name
\large \today\\[2cm] % Date


\end{center}

\end{titlepage}

%----------------------------------------------------------------------------------------
%	DECLARATION PAGE
%	Your institution may give you a different text to place here
%----------------------------------------------------------------------------------------

\Declaration{\addtocontents{toc}{\vspace{1em}}} % Add a gap in the Contents, for aesthetics
\setcounter{page}{2}

\begin{minipage}{1.0\textwidth}
    
    It is certified that the work contained in this thesis entitled \textbf{\enquote{\ttitle}} by \textbf{\authornames} has been carried out under my supervision and that it has not been submitted elsewhere for a degree.
        
\end{minipage}

\vspace{45mm}

\begin{tabular}{@{}p{0.465\textwidth}@{}p{0.535\textwidth}@{}}
    \textbf{Prof. Vishal Agarwal} & \textbf{Prof. Swarnendu Biswas} \\
    Associate Professor & Assistant Professor \\
    Department of Chemical Engineering & Department of Computer Science \& Engineering \\
    Indian Institute of Technology Kanpur & Indian Institute of Technology Kanpur \\
\end{tabular}

\vfill
\clearpage % Start a new page
\StudentDeclaration{\addtocontents{toc}{\vspace{1em}}} % Add a gap in the Contents, for aesthetics

This is to certify that the thesis titled \textbf{``\ttitle''} has been authored by me. It presents the research conducted by me under the supervision of \textbf{\supnameA} and \textbf{\supnameB}.\par

To the best of my knowledge, it is an original work, both in terms of research content and narrative, and has not been submitted for a degree elsewhere, in part or in full. Further, due credit has been attributed to the relevant state-of-the-art collaborations with appropriate citations and acknowledgements, which are in line with established norms and practices.\\ [2cm]
\begin{minipage}{.5\textwidth}
		\begin{flushleft}
			{\authornames\\ Roll No. 208070710 \\
			\normalsize{\href{http://www.iitk.ac.in/che}{Department of Chemical Engineering}\\
			\univname}}
		\end{flushleft}
\end{minipage}
\vfill

\clearpage % Start a new page
%----------------------------------------------------------------------------------------
%	ABSTRACT PAGE
%----------------------------------------------------------------------------------------

%\addtotoc{Abstract} % Add the "Abstract" page entry to the Contents
\lhead{\emph{Abstract}}

\abstract{\addtocontents{toc}{\vspace{1em}} % Add a gap in the Contents, for aesthetics

Accurate computation of long-range coulombic interactions is essential in molecular simulations, as they dominate other interactions, such as van der Waals. However, calculating these interactions can be computationally expensive, particularly for large-scale systems. Large-scale systems are typically modelled as a small simulation cell periodically repeated in all three dimensions. The traditional Ewald summation~\cite{Ewald1921, Brooks1989} addresses the challenge of computing Coulombic interaction with complete 3D Periodicity. It splits the Coulomb potential into two parts: a short-range component handled in real space and a long-range component in reciprocal space using Fourier transforms. The short-range part rapidly converges, while the long-range part uses Fast Fourier Transform (FFT) and interpolation techniques, which scale as O($N\log N$) time complexity, where $N$ is the number of charged ions.

The Ewald method becomes more challenging for systems with reduced periodicity, such as 2D slab geometries. These systems are periodic in two dimensions and finite in the third, which disturbs the overall symmetry essential for fast computation of Ewald summation. Traditional or modern approaches struggle with computational efficiency and accuracy for slab geometries. These methods often require complex corrections or approximations that render the simulations slower than 3D Ewald summation. 

In this work, we propose a method of computing an exact 2D Ewald summation, which is more efficient than the 3D Ewald summation. Our approach utilizes a modified screening function for the reciprocal space summation within the Particle-Mesh Ewald (PME) method. Using FFT for reciprocal space computations, our method achieves the same computational complexity, O($N\log N$), as the traditional 3D Ewald technique. 

We also carried out a performance analysis of codes using the Intel\textsuperscript{\textregistered} VTune framework to identify performance bottlenecks and guide the optimization efforts. Special polynomial approximations were introduced for the real-space energy calculation to reduce the computational overheads without sacrificing accuracy. To further enhance the performance on multicore architectures, we parallelized the program using OpenMP, achieving a significant reduction in the runtime.@@@Rewrite@@@

These improvements provide a more efficient framework for simulating 2D periodic systems, significantly reducing computational expenses. By maintaining accuracy without compromising computational speed, our method offers a practical and efficient framework for large-scale molecular simulations, which will have implications for simulating 2D periodic systems found in surface science, material interfaces, and thin films.

}

%----------------------------------------------------------------------------------------
%	Declaration Page
%----------------------------------------------------------------------------------------



%----------------------------------------------------------------------------------------
%	ACKNOWLEDGEMENTS
%----------------------------------------------------------------------------------------
\clearpage % Start a new page
\setstretch{1.3} % Reset the line-spacing to 1.3 for body text (if it has changed)
\lhead{\emph{Acknowledgements}}
\acknowledgements{\addtocontents{toc}{\vspace{1em}} % Add a gap in the Contents, for aesthetics

Acknowledgements

}
\clearpage % Start a new page

%----------------------------------------------------------------------------------------
%	LIST OF CONTENTS/FIGURES/TABLES PAGES
%----------------------------------------------------------------------------------------

\pagestyle{fancy} % The page style headers have been "empty" all this time, now use the "fancy" headers as defined before to bring them back

\lhead{\emph{Contents}} % Set the left side page header to "Contents"
\tableofcontents % Write out the Table of Contents

\lhead{\emph{List of Figures}} % Set the left side page header to "List of Figures"
\listoffigures % Write out the List of Figures

\lhead{\emph{List of Tables}} % Set the left side page header to "List of Tables"
\listoftables % Write out the List of Tables

%----------------------------------------------------------------------------------------
%	ABBREVIATIONS
%----------------------------------------------------------------------------------------

\clearpage % Start a new page

\setstretch{1.5} % Set the line spacing to 1.5, this makes the following tables easier to read

\lhead{\emph{Abbreviations}} % Set the left side page header to "Abbreviations"

\chapter*{Abbreviations}
\addtotoc{Abbreviations}
\begin{acronym}[XXXXXXXXX] % Give the longest label here so that the list is nicely aligned
\acro{EBSD}{electron backscatter diffraction}
\acro{PME}{Particle Mesh Ewald}
\acro{SPME}{Particle Mesh Ewald}

\end{acronym}

%----------------------------------------------------------------------------------------
%	PHYSICAL CONSTANTS/OTHER DEFINITIONS
%----------------------------------------------------------------------------------------
%
\clearpage % Start a new page

\lhead{\emph{Physical Constants}} % Set the left side page header to "Physical Constants"

\listofconstants{lrcl} % Include a list of Physical Constants (a four column table)
{
Speed of Light & $c$ & $=$ & $2.997\ 924\ 58\times10^{8}\ \mbox{ms}^{-\mbox{s}}$ (exact)\\
% Constant Name & Symbol & = & Constant Value (with units) \\
}

%----------------------------------------------------------------------------------------
%	SYMBOLS
%----------------------------------------------------------------------------------------

\clearpage % Start a new page

\lhead{\emph{Symbols}} % Set the left side page header to "Symbols"

\listofnomenclature{lll} % Include a list of Symbols (a two column table)
{
% Symbol & Name & Unit \\
$C_g$ & contiguity \\
$\phi_d$ & Dihedral angle \\
$U^{LR}$ & Ewald Energy for Reciprocal (Long-Range) Space \\ 
$U^{SR}$ & Ewald Energy for Real (Short-Range) Space \\ 
$U^{S}$ & Self Interaction Correction to Ewald Energy \\ 
}

% \ListofPublications{\addtocontents{toc}{\vspace{1em}} % Add a gap in the Contents, for aesthetics

% \textbf{Publications from Thesis}
% \begin{enumerate}

%     \item Paper 1  \href{https://doi.org/10.1016/j.ijrmhm.2022.105849}{\textit{\textcolor{blue}{10.1016/j.ijrmhm.2022.105849}}}.
    
%     \item Paper 2. \href{https://doi.org/10.1016/j.matchar.2022.112010}{\textit{\textcolor{blue}{10.1016/j.matchar.2022.112010}}}.
    
%     \item Paper 3. \href{https://doi.org/10.1007/s11661-021-06586-x}{\textit{\textcolor{blue}{10.1007/s11661-021-06586-x}}}.
%     \end{enumerate}
    
% \textbf{Others}

% \begin{enumerate}
%     \item Paper 4 \href{https://doi.org/10.1080/02670836.2021.2007455}{\textit{\textcolor{blue}{10.1080/02670836.2021.2007455}}}.
    
%     \item Paper 5 \href{https://doi.org/10.1080/02670836.2021.1946949}{\textit{\textcolor{blue}{10.1080/02670836.2021.1946949}}}.
    
%     \item Paper 6 \href{https://doi.org/10.1080/02670836.2020.1773036}{\textit{\textcolor{blue}{10.1080/02670836.2020.1773036}}}.
    
%     \item Paper 7 \href{https://doi.org/10.2139/ssrn.4125910}{\textit{\textcolor{blue}{10.2139/ssrn.4125910}}}.
    
% \end{enumerate}
% }

\clearpage % Start a new page


\setstretch{1.3} % Return the line spacing back to 1.3
%
\pagestyle{empty} % Page style needs to be empty for this page
%
\dedicatory{Dedicated to TBD} % Dedication text
%
\addtocontents{toc}{\vspace{2em}} % Add a gap in the Contents, for aesthetics

%----------------------------------------------------------------------------------------
%	THESIS CONTENT - CHAPTERS
%----------------------------------------------------------------------------------------

\mainmatter % Begin numeric (1,2,3...) page numbering

\pagestyle{fancy} % Return the page headers back to the "fancy" style

% Include the chapters of the thesis as separate files from the Chapters folder
% Uncomment the lines as you write the chapters
\clearpage
% Chapter Template

\chapter{Introduction} % Main chapter title

\label{Chapter1}

\lhead{Chapter 1. \emph{Introduction}} % Change X to a consecutive number; this is for the header on each page - perhaps a shortened title

%----------------------------------------------------------------------------------------
%	SECTION 1
%----------------------------------------------------------------------------------------

\section{Background}
\lipsum[2]
\section{Motivation}

\lipsum[2]

\section{Organization of the Thesis}

This is how acronym is added \ac{EBSD}. The present dissertation is divided into eight chapters, each of which is further divided into well-structured sections and subsections. \autoref{Chapter1} explains the rationale behind the dissertation and its objectives. \autoref{Chapter2} says something more. 

% Chapter Template

\chapter{Literature Review}

\label{Chapter2}

\lhead{Chapter 2. \emph{Literature Review}} 

\section{Heavy Alloy Production and Properties}
\label{Heavy Alloy Production and Properties}
\lipsum[2]

\begin{figure}[H]
    \centering
    \includegraphics[width=0.5\textwidth]{images/Figure 1.png}
    \caption{This is the caption for figure 1.}
    \label{figure:chap2_Figure_2}
\end{figure}
\input{Chapters/Chapter3}
% Chapter Template

\chapter{New Algorithm/ Our Work}

\label{Chapter4}

\lhead{Chapter 4. \emph{New Algorithm/ Our Work}} 
In the case of a two-dimensional periodic system, the summation over the $m_z$ component in Eq.~(\ref{eq:coul}) is omitted, which breaks the symmetry present in the fully periodic (3D) formulation. However, we aim to retain the $m_z$ summation, similar to the three-dimensional periodic system, by introducing a top-hat function. This function is designed to eliminate the unwanted contributions from periodic images along the $z$-direction, treating these images as non-existent.
\section{Top-hat function}
This work introduces a vacuum within our unit cell along the $z$-axis, resulting in a total length of $ L_z$  in this direction. This length must satisfy the condition $L_z \geq 2L$, where $L$ is the side length of the original simulation box.
This condition ensures that the vacuum region is large enough to prevent atoms from adjacent periodic images from interacting with the original cell. Such interactions can lead to artificial periodic effects that would compromise the accuracy of the simulation.

\begin{figure}[]% if you want to fix the position of this image, add H inside the []
    \centering
    \includegraphics[width=\linewidth]{images/simulationcell.jpg}
    \caption{insert some caption}
    \label{fig:simulation_cell}
\end{figure}

To exclude the system's interactions along the $z$-direction, a spatially dependent window function $\phi(z)$ is introduced. The simplest version of $\phi(z)$ is a top-hat function defined as:
\begin{flalign*} 
    \phi(z) = 
    \begin{cases} 
        1 & \text{if } z \in (-\frac{L_z}{2}, \frac{L_z}{2}) \\
        0 & \text{otherwise}
    \end{cases}
\end{flalign*}

While this function is straightforward, its sharp discontinuities at the edges can introduce numerical artefacts, especially in methods sensitive to derivatives or smoothness (e.g., force field calculations). To address this, we replace the discontinuous top-hat function with a smooth approximation that retains the essential features of the original function while ensuring a differentiable and numerically stable function.
The smoothed version of $\phi(z)$ is expressed as:
\begin{flalign}
    \phi(z) &= \frac{1}{1+ exp(-\gamma(0.5L_z+z))} + \frac{1}{1+ exp(-\gamma(0.5L_z-z))} -1 \label{eq:1}
\end{flalign}

This function uses a linear combination of two sigmoid functions to make a symmetrical top-hat function about the origin. The positive constant \textbf{$\gamma$} determines the steepness of the top-hat around the edges. The behaviour of $\phi(z)$ eq.(\ref{eq:1}) is shown in Figure \ref{fig:tophat}.
Other forms of the top-hat functions can also be used; some examples are as follows:
\begin{flalign}
        \phi(z) &= \frac{1}{2}[ \tanh(\gamma(x + 0.5L)) -\tanh(\gamma(x - 0.5L)) ] \\
        \phi(z) &= \frac{2}{\pi}\left[ \arctan \left( e^{\gamma (-x + 0.5L)} \right) + \arctan \left( e^{\gamma (x + 0.5L)} \right) \right]- 1 
\end{flalign}
\begin{figure}[htbp]
    \centering
    \includegraphics[width=0.7\linewidth]{images/TopHat2.png}
    \caption{Visualization of top-hat function for different $\gamma$. As $\gamma$ increases, the function transitions more sharply at the boundaries, approaching an ideal window function.}
    % \parbox{0.5\linewidth}{\justifying \caption{Visualization of top-hat function for different $\gamma$. As $\gamma$ increases, the function transitions more sharply at the boundaries, approaching an ideal window function.}}
    \label{fig:tophat}
\end{figure}
\section{Incorporating the top-hat function}
To evaluate the long-range contribution to the electrostatic energy in a two-dimensional (2D) periodic system, the derivation begins with the standard three-dimensional (3D) Ewald summation expression. A top-hat function, denoted by $\phi(z)$, is introduced to eliminate the influence of periodic images along the non-periodic $z$-direction. This function ensures that only the real physical interactions in the $z$-direction are retained, effectively reducing the system to 2D while preserving the 3D summation structure.

The long-range part of the Coulomb interaction, initially expressed as $\frac{\text{erf}(\alpha r)}{r}$, is converted into an integral form using the identity:
$$
\frac{\text{erf}(\alpha r)}{r} = \frac{2}{\sqrt{\pi}} \int_0^\alpha e^{-r^2 t^2} dt.
$$

This transformation separates the exponential term into components along the $x$, $y$, and $z$ directions.
\begin{flalign}
    \nonumber(4\pi\epsilon_o)U^{LR} & = \frac{1}{2}\sum_{\vec{M}=-\infty}^{\infty}\sum_{i=1}^{n_p}\sum_{j=1}^{n_p}q_i q_j \frac{{erf}(\alpha|\vec{r_i}-\vec{r_j}+\vec{M}.\vec{L}|)}{|\vec{r_i}-\vec{r_j}+\vec{M}.\vec{L}|} \,\phi(z_{ij}+m_zL_z)\quad\quad\quad\quad
\end{flalign}
\begin{flalign}
    \nonumber\quad\quad\quad\quad\quad& = \frac{1}{2}\sum_{\vec{M}=-\infty}^{\infty}\sum_{i=1}^{n_p}\sum_{j=1}^{n_p}\frac{2q_i q_j}{\sqrt{\pi}}\int_{0}^{\alpha}dt\,  {exp}(-|\vec{r_i}-\vec{r_j}+\vec{M}.\vec{L}|^2 t^2)\,\phi(z_{ij}+m_zL_z)\\
    \nonumber &=\frac{1}{2}\sum_{\vec{M}=-\infty}^{\infty}\sum_{i=1}^{n_p}\sum_{j=1}^{n_p}\frac{2q_i q_j}{\sqrt{\pi}}\int_{0}^{\alpha}dt\,  {exp}\left[-(x_{ij}+m_xL_x)^2 t^2\right] \\
    &\quad\quad\quad\times{exp}\left[-(y_{ij}+m_yL_y)^2 t^2\right]\times{exp}\left[-(z_{ij}+m_zL_z)^2 t^2\right] \, \phi(z_{ij}+m_zL_z)
\end{flalign}

The Poisson summation formula is used to convert the expression from real space to reciprocal space. The detailed derivation of this transformation is provided in the \colorbox{yellow}{Appendix}.
\begin{flalign}
    \nonumber(4\pi\epsilon_o)U^{LR}& =\frac{\sqrt{\pi}}{L_xL_y}\sum_{\vec{\mathbf{k}}=-\infty}^{\infty}\sum_{j=1}^{n_p}\sum_{k=1}^{n_p}q_j q_k \int_{0}^{\alpha}\frac{dt}{t^2}\,{exp}\left(\frac{-\pi^2k_x^2}{L_x^2t^2}+i \frac{2\pi k_xx_{ij}}{L_x}\right)
    \\&\quad\quad\quad
    \times\,{exp}\left(\frac{-\pi^2k_y^2}{L_y^2t^2}+i\frac{2\pi k_yy_{ij}}{L_y}\right)\times C_{k_z}(t)\,{ exp}\left(i\frac{2\pi k_zz_{ij}}{L_z}\right)
\end{flalign}

The term $C_{k_z}(t)$ is expressed as follows:
\begin{flalign}
     C_{k_z}(t) &=\frac{1}{L_z}\int_{-\infty}^{\infty}ds\,exp(-i\frac{2\pi n s}{L_z})\, exp(-s^2t^2)\, \phi(s) \label{eq:Cz}
     % \\ &=\frac{1}{L_z}\int_{-\infty}^{\infty}ds\hspace{1mm}exp(-i\frac{2\pi n s}{L_z})\hspace{1mm} exp(-s^2t^2)\left[\frac{1}{1+ exp(-\gamma(0.5L_z+s))} + \frac{1}{1+ exp(-\gamma(0.5L_z-s))} -1\right]
\end{flalign}
The summation is expressed using the reciprocal vector formulation as
\begin{flalign*}
    \nonumber(4\pi\epsilon_o)U^{LR}& =\frac{\sqrt{\pi}}{L_xL_y}\sum_{\vec{\mathbf{k}}=-\infty}^{\infty}\sum_{i=1}^{n_p}\sum_{j=1}^{n_p}q_i q_j\int_{0}^{\alpha}\frac{dt}{t^2}\,{ exp}\left[i\vec G.(\vec r_i-\vec r_j)\right]\times C_{k_z}(t)\,{ exp}\left(\frac{-1}{4t^2}|\vec G|_{xy}^2\right)\\
    &=\frac{\sqrt{\pi}}{L_xL_y}\sum_{\vec{\mathbf{k}}=-\infty}^{\infty}\left[ \int_{0}^{\alpha}\frac{dt}{t^2}C_{k_z}(t)\,{exp}\left(\frac{-1}{4t^2}|\vec G|_{xy}^2\right)\right]\sum_{j=1}^{n_p}\sum_{k=1}^{n_p}q_i q_j\,{ exp}\left[i\vec G.(\vec r_i-\vec r_j)\right]
\end{flalign*}
The double summation over indices $j$ and $i$ is reformulated as a single summation incorporating the structure factor Eq.(\ref{eq:structurefactor}).
% \begin{flalign*}
    % S(\vec G)&=\sum_{j=1}^{n_p}q_j\,exp(i\vec G.\vec r_j)
% \end{flalign*}
Hence, the final expression for the two-dimensional Ewald summation takes the form
\begin{flalign}
    (4\pi\epsilon_o)U^{LR}& =\frac{\sqrt{\pi}}{L_xL_y}\sum_{\vec{\mathbf{k}}=-\infty}^{\infty}{}^\prime\left[ \int_{0}^{\alpha}\frac{dt}{t^2}C_{k_z}(t){exp}\left(\frac{-1}{4t^2}|\vec G|_{xy}^2\right)\right] |\,S(\vec G)\,|^2
\end{flalign}

The prime symbol (${}^\prime$) indicates that the $\vec{k} = 0$ term is excluded from the summation. This term does not contribute to the result when the system is electrically neutral, i.e., when the total charge sums to zero. 
Define $\chi(p_1,p_2,p_3)_{2D-Modified}$ as
\begin{flalign}
    \nonumber \chi(p_x,p_y,p_z)_{2D-Modified} &= \frac{\sqrt{\pi}}{L_xL_y}\sum_{\vec{\mathbf{k}}=-\infty}^{\infty}{}^\prime \left[ \int_{0}^{\alpha}\frac{dt}{t^2}C_{k_z}(t){exp}\left(\frac{-1}{4t^2}|\vec G|_{xy}^2\right)\right] 
    \\&\quad\quad\quad\times \exp\left( i \frac{2\pi k_x p_x}{K_x} + i \frac{2\pi k_y p_y}{K_y} + i \frac{2\pi k_z p_z}{K_z} \right)
    \\&=\mathcal{F}_{3D}(E)(p_x,p_y,p_z)
\end{flalign}
where the new screening factor array $E$ is defined as
\begin{flalign}
    E(k_x,k_y,k_z) &= \frac{\sqrt{\pi}}{L_xL_y}\left[ \int_{0}^{\alpha}\frac{dt}{t^2}C_{k_z}(t){exp}\left(\frac{-1}{4t^2}|\vec G|_{xy}^2\right)\right] 
\end{flalign}
Note that $E = \mathcal{F}^{-1}_{3D}(\chi)$. The expression within the brackets $[\ \ ]$ is independent of the orientation or distribution of the particles in the system. Therefore, it can be computed in advance, prior to the simulation, without affecting the overall computational cost.

Furthermore, the structure factor can be efficiently calculated using the Smooth Particle Mesh Ewald (SPME) method proposed by Essmann et al., by applying Fast Fourier Transforms (FFTs).
\begin{flalign}
    \nonumber (4\pi\epsilon_o)U^{LR}  & \approx \frac{\sqrt{\pi}}{L_xL_y}\sum_{\vec{\mathbf{k}}=-\infty}^{\infty}{}^{\prime}\left[ \int_{0}^{\alpha}\frac{dt}{t^2}C_{k_z}(t){exp}\left(\frac{-1}{4t^2}|\vec G|_{xy}^2\right)\right]\,
    \\& \quad\quad B_{3D}(k_x,k_y,k_z) \left|\mathcal{F}_{3D}(Q)(k_x,k_y,k_z)\right|^2 \label{eq:newreci2DSPME}
\end{flalign}

\section{Choosing Parameters}
\subsection{Selection of $\gamma$}
The Fourier transform of a top-hat function is known to yield oscillatory output. 
% It is hypothesised that these oscillations contribute to the instability of the method, thus hindering the goal of reducing error rates. 
This section seeks to identify suitable values of $\gamma$ that can reduce these oscillations.

In particular, it is noted that the expression in Eq.~(\ref{eq:Cz}) represents the Fourier transform of the product of a top-hat function, $\phi(s)$, and a Gaussian function.It is hypothesised that the majority of the oscillations in the calculations would arise from the top-hat component. To investigate this, the behaviour of its Fourier transform is examined by varying the parameters $\gamma$ and $L_z$, with the goal of determining configurations that minimise the undesirable oscillatory effects.

In Fig.~\ref{fig:tophat}, as the parameter $\gamma$ decreases, the top-hat function becomes increasingly localised and narrower in real space. Conversely, larger values of $\gamma$ produce a broader profile. However, as shown in Fig.~\ref{fig:fourieroftophatvarygammaL300}, broader top-hat functions lead to extended and persistent oscillations in the frequency domain, which can adversely affect the convergence of numerical computations for reciprocal space energies. To mitigate these oscillations, smaller values of $\gamma$ are preferred, as they result in narrower real-space functions with more rapidly decaying Fourier transforms. The trade-off, however, is that a smaller $\gamma$ reduces the region over which the top-hat function maintains a value of exactly one. To accommodate this, the domain length $L_z$ must be increased to ensure the simulation box remains fully within this central region.

\begin{figure}[htbp]
  \centering
  \includegraphics[width=\linewidth]{images/fourieroftophatvarygammaL300.jpg}
  \caption{Fourier Transform of Top-hat Function with Varying $\gamma$, $L_z = 300$}
  \label{fig:fourieroftophatvarygammaL300}
\end{figure}
  
  % \vspace{1em} % Optional spacing between figures

\begin{figure}[htbp]
  \centering
  \includegraphics[width=\linewidth]{images/fourieroftophatvaryL_gamma0.5.jpg}
  \caption{Fourier Transform of Top-hat Function with Varying $L_z$, $\gamma = 0.5$}
  \label{fig:fourieroftophatvaryL_gamma0}
\end{figure}


Fig.~\ref{fig:fourieroftophatvaryL_gamma0} supports this choice by showing the Fourier Transform for varying $L_z$ at a fixed $\gamma = 0.5$. It demonstrates that changing $L_z$ does not impact the convergence or magnitude of the oscillations in the frequency domain. Although the damping frequency, where oscillations begin to decay, shifts higher with larger $L_z$, this does not affect the essential behaviour of the transform. This confirms that increasing $L_z$ when using smaller $\gamma$ values does not introduce computational issues, making this a suitable strategy for reducing oscillations while preserving accuracy.

\subsection{Selection of  $L_z$}
To ensure that the simulation cell is completely contained within the region where $\phi(z) = 1$, the smallest possible value for $L_z$ is selected. This approach not only guarantees the proper placement of the simulation cell but also reduces the number of grid points required for the Smooth Particle Mesh Ewald (SPME) method, thereby enhancing computational efficiency. To ascertain the optimal value of $L_z$, a binary search-based algorithm (refer to \textbf{Alg. \ref{alg:vacuum}}) was implemented to determine the box length.
\input{binaryAlgorithm}
\chapter{Implementation, Performance Analysis and Optimization}
\label{Chapter5}
\lhead{Chapter 5. \emph{Implementation, Performance Analysis, and Optimization}}

\section{Algorithmic Overview and Pseudocode}

% \swb{The following paragraph should go in Section 4.2. Only write about the algorithm here.}
In this section, we will elaborate on the logical flow of the program for our method. For complete code, refer to Appendix \ref{AppendixB}. 
\subsection{Important Helper Functions}
\textbf{Screening Factor Array}: This function computes the product of the screening factor array defined in Eq.~(\ref{eq:N2D-ScreenFunction}) and the B-spline coefficient array defined in Eq.~(\ref{eq:bsplineArray}). This array can be precomputed and stored in memory prior to running the simulation, allowing it to be reused throughout the runs.
\lstinputlisting[language=C]{CodeFiles/screenfunction.c}

\textbf{Implementing \ac{MIC}}: This function calculates the distance between two atoms using the \aclu{MIC} (\ac{MIC}) for system with 2D periodicity.
\lstinputlisting[language=C]{CodeFiles/mic.c}

\subsection{Functions to Compute Energies}
\textbf{Real Space Energy}: The loops in this function computes the total real space energy. 
\lstinputlisting[language=C]{CodeFiles/real.c}

\textbf{Reciprocal Space Energy with Particle Mesh Adaptation}: The loops in this function computes the total reciprocal space energy. 
\lstinputlisting[language=C]{CodeFiles/reci.c}

\section{Analysis of Baseline Program}

After implementing the new 2D Ewald summation algorithm, the subsequent objective was to identify performance bottlenecks in the program, as the next critical step in guiding further optimization efforts. Intel\textsuperscript{\textregistered} VTune\texttrademark{} Profiler was used for this purpose. VTune supports a wide range of analysis types, including Hotspots, Microarchitecture Exploration, Threading, Memory Access, and Platform analysis. With support for multiple programming languages and parallelization frameworks like OpenMP and MPI, VTune improves application speed, scalability, and hardware efficiency across diverse Intel architectures. 

The Hotspot analysis is particularly useful in the early development process of an application as it helps pinpoint the sections of code where the CPU spends the most time, guiding the optimization efforts for maximum performance gains. In the Bottom-up window, detailed performance data is grouped by function and sorted by CPU time. Functions with high CPU time should be prioritized for optimization.

% \swb{Write a few more lines justifying our choice with VTune. Also, list a few different program analyses that VTune supports. Tell readers what is analysed in hotspot analysis and what is the expected output.}

% \swb{It seems we need to briefly describe the implementation before we talk about profiling it.}

Hotspot analysis of the baseline program was performed to identify computational bottlenecks. The results in Fig.~(\ref{fig:result1vtune}), indicate that the real-space component accounts for approximately 60-65\% of the total CPU time. The most expensive function was \texttt{dist}, responsible for inter-particle distance calculations, consuming around 25\% of the execution time. This cost is compounded by repeated calls to the \texttt{\_\_erfc} function, contributing approximately 20\% of the total runtime. These findings highlight the need to optimize the real-space calculations, particularly the error function evaluation, to improve overall performance.

While both \texttt{dist} and \texttt{\_\_erfc} are major contributors to the CPU Time, the \texttt{dist} function offers limited scope for optimization due to the straightforward nature of the algorithm. Therefore, the focus shifts to the \texttt{\_\_erfc} function from \texttt{libm}, the standard math library in the GNU C Library. Using a faster or approximate version of the error function could help reduce execution time and improve performance.

% \swb{What is \texttt{libm}?}

% \begin{figure}[htbp]
\begin{figure}[H]
    \centering
    \begin{minipage}{0.7\textwidth}
        \fbox{\includegraphics[width=\linewidth]{images/VTuneInitialTime.png}}
    \end{minipage}%
    \begin{minipage}{0.3\textwidth}
        \caption{Execution time details for baseline program.}
    \end{minipage}
\end{figure}
% \begin{figure}[htbp]
\begin{figure}[h]
    \centering
    \includegraphics[width = \linewidth]{images/VTuneInitialHotspot.png}
    \caption{Hotspot analysis of the baseline Ewald summation implementation, as reported by Intel VTune Profiler. A significant portion of total CPU time is concentrated in the \textit{dist} function (25.8\%), the standard math library's \textit{\_\_erfc} function (19.4\%).}
    \label{fig:result1vtune}
\end{figure}

\section{Polynomial Interpolation Optimization}
\subsection{Polynomial Interpolation of Error Function}
As deduced in the previous section, a significant portion of the program’s execution time was consumed by calls to the \verb|std::erfc| function. To reduce the computational cost associated with evaluating the complementary error function \(\operatorname{erfc}(x)\) in the real-space part of the Ewald summation, a polynomial interpolation approach was adopted. This technique is outlined in \textit{Handbook of Mathematical Functions by Abramowitz and Stegun}~\cite{abramowitz1965handbook}. The expression used is
\begin{equation}  
    erf(x) = 1 - (a_1 t + a_2 t^2 + a_3 t^3 + a_4 t^4 + a_5 t^5) e^{-x^2} + \epsilon(x)
\end{equation}
\[
    t = \frac{1}{1 + px}, \quad |\epsilon(x)| \leq 1.5 \times 10^{-7},
\]
\[
p = 0.3275911, \\
a_1 = 0.254829592, \\
a_2 = -0.284496736,
\]
\[
a_3 = 1.421413741, \\
a_4 = -1.453152027, \\
a_5 = 1.061405429.
\]
The polynomial was subsequently adapted as a replacement for the standard $\operatorname{erfc}(x)$ function in the real-space evaluation. This approach is advantageous because $\operatorname{erfc}(x)$ is computationally more expensive than $\operatorname{exp}(x)$. The evaluation of $\operatorname{erfc}(x)$ requires complex numerical approximations to compute an integral that does not have a simple closed form. In contrast, $\operatorname{exp}(x)$ is simpler and optimized, which makes it significantly faster to compute.

\subsection{Efficiency and Accuracy}
In order to assess the efficiency of the interpolation-based approach, a series of computations were performed. The number of ions in the system was varied while keeping the box side constant at 25 \AA. 

As shown in the Fig.~(\ref{fig:realspaceopt}), this numerical optimization resulted in a significant improvement in performance. On an average, the interpolation method achieved a $15\%$ reduction in runtime, showing improved scalability and efficiency, especially for larger system sizes. This speed-up was achieved without compromising numerical accuracy; the maximum relative error in the real space energies was less than $10^{-5}$, which is well within the acceptable bounds for energy calculations in molecular simulations.
\begin{figure}[H]
    \centering
    \includegraphics[width=0.75\linewidth]{images/realspaceopt.jpg}
    \caption{Comparison of CPU runtime between the polynomial approximation and exact error function evaluation to compute real space energy as a function of the number of atoms. The side length of the simulation box was fixed at 25 \AA. The polynomial approximation method demonstrates a consistent speed-up of approximately $15\%$ across all system sizes, while maintaining relative errors under $10^{-5}$}
    \label{fig:realspaceopt}
\end{figure}

\section{Code-Level Optimizations}
\subsection{Array Flattening}
In the implementation, data involving multiple dimensions must be stored in several parts of the program, such as atom positions and charge spreading array for the SPME. While multidimensional arrays are a natural choice for such data, they have drawbacks. 

Dynamically allocated multidimensional arrays often lead to scattered memory layouts and multiple pointer dereferences. This results in poor cache performance and added complexity in memory management. 
To address this, a one-dimensional array was used to represent the multidimensional structure. For an array with rank $d$ and dimensions $n_1\times \ldots \times n_d$, an element at ($i_1\times \ldots \times i_d$) maps to:
\begin{flalign*}
    i_d + n_d \cdot \left( i_{d-1} + n_{d-1} \cdot \left( \ldots + n_2 \cdot i_1 \right)\right)
\end{flalign*}
This approach reduced overhead, improved memory locality, and allowed faster access through direct indexing. %%In the paper we would want to elaborate this...
\subsection{Vectorization}
Vectorization enables the simultaneous processing of multiple data elements using a single instruction. This approach significantly improves both the speed and efficiency of computations. It is a fundamental technique in high-performance numerical computing, scientific simulations, graphics, and machine learning, as it leverages the SIMD (Single Instruction Multiple Data)~\cite{hennessy2017computer} capabilities present in modern processors.

In this work, the program has been compiled using the flags \texttt{-O3}, \texttt{-mavx2}, \texttt{-march=native}, \texttt{-ftree-vectorize}, and \texttt{-ftree-vectorizer-verbose=1}. The \texttt{-O3} flag enables aggressive optimization, including automatic vectorization of loops. The \texttt{-mavx2} flag ensures that the generated code utilizes AVX2 instructions, which operate on 256-bit registers. This allows the simultaneous processing of 8 single-precision floating-point numbers or 4 double-precision floating-point numbers, thereby greatly accelerating loops and computation-intensive sections of the code.


\section{Profiling of Optimized Program}
% \section{Performance of Optimized Implementation}
Following the improvements, hotspot analysis of the optimized implementation Fig.~(\ref{fig:resultVTuneFinal}), shows a notable shift in the computational profile. The \texttt{\_\_erfc} function no longer appears among the major hotspots. The primary contributors to CPU time are now \texttt{dist} (35.4\%) and \texttt{real\_omp\_fn.0} (23.2\%), both associated with real-space computations.  %%%For the paper, could we give a reason that paralellization can improve this bottleneck. 
% Detailed performance improvements are presented in the \textit{Numerical Analysis} section of the thesis.
\begin{figure}[htbp]
% \begin{figure}[]
    \centering
    \includegraphics[width = \linewidth]{images/VTuneFinalHotSpots.png}
    \caption{Hotspot analysis of the optimized Ewald summation implementation, using a polynomial interpolation for the error function. The overall CPU time distribution indicates improved efficiency in the real-space term.}
    \label{fig:resultVTuneFinal}
\end{figure}
\begin{figure}[h]
% \begin{figure}[]
    \centering
    \begin{minipage}{0.7\textwidth}
        \fbox{\includegraphics[width=\linewidth]{images/VTuneFinalTime.png}}
    \end{minipage}%
    \begin{minipage}{0.3\textwidth}
        \caption{Execution time details for improved program.}
    \end{minipage}
\end{figure}

\section{Parallelization}
Parallel programming can have an enormous impact on the performance and scalability of computational applications~\cite{pacheco2011introduction}. The motivation for parallelizing code is to reduce its execution time, enabling it to run faster on modern multiprocessor systems. In this context, the Ewald summation algorithm is crucial, as it is computationally intensive. Depending on the specific component of the calculation, its computational complexity ranges from $O(N^2)$ to $O(NlogN)$, making it essential to accelerate through parallel techniques. Efficient parallelization is critical for simulating long-range interactions in order to study large-scale systems.

\subsection{OpenMP}
In the Ewald summation method, a significant portion of the execution time is consumed by for loops. These loops are particularly well-suited for parallelization using OpenMP~\cite{dagum1998openmp}, a shared-memory parallel programming model that efficiently distributes computational workloads across multiple threads, thereby enhancing overall performance.

% \swb{Cite OpenMP.}

OpenMP facilitates incremental parallelization of existing sequential code with minimal modifications. Its accessibility and simplicity make it especially advantageous, as it does not require specialized hardware such as supercomputers. Instead, it can yield substantial performance improvements even on standard personal computers equipped with multi-core processors.

OpenMP utilizes \verb|#pragma|  directives, which are combined with directive specifications and optional clauses. An OpenMP directive begins with \verb|#pragma omp|, followed by a specific directive keyword. 
By default, OpenMP creates a number of threads that is typically equal to the number of CPU cores available. However, this can also be modified via \verb|omp_set_num_threads()| function. For instance, the parallelization of a for loop is expressed as follows:
\lstinputlisting[language=C]{CodeFiles/forloop.c}

% \swb{Did you mention how many threads are created by default?}

\subsection{Ensuring Correctness in Loop-Carried Dependencies}
\textbf{Reduction:} The \verb|reduction| clause is used to safely accumulate the result of a shared variable such as the real or the reciprocal energies. Each thread maintains a private copy of the reduction variable, and they automatically combine that at the end of the parallel region using the specified reduction operator. An example of its application in the reciprocal space energy calculation is shown below
\lstinputlisting[language=C]{CodeFiles/reduction.c}

\textbf{Race condition during charge spreading:} A naïve approach that assigns one thread to map each charge onto the grid can lead to synchronization issues when multiple threads attempt to update the same grid point. During the computation of the charge spreading array $Q$, overlapping interpolation regions may cause several threads to modify the same grid location at the same time, resulting in a race condition and potentially incorrect values. To handle this, the \verb|atomic update| clause in OpenMP was used to ensure that updates are performed atomically, thereby preventing simultaneous read and write operations by different threads.
\lstinputlisting[language=C]{CodeFiles/atomic.c}

\textbf{Nested loops:} To parallelize the independent nested loops involved in the reciprocal space summation over the three-dimensional grids, the \texttt{collapse(3)} clause is employed. This directive flattens the nested loops into a single iteration space, thereby enhancing load balancing and facilitating uniform distribution of the computational workload across threads.
\lstinputlisting[language=C]{CodeFiles/collapse.c}
% \clearpage
\subsection{Performance Evaluation}
To evaluate the effectiveness of OpenMP-based parallelization, a series of calculations were performed across different system sizes. The execution times were recorded by varying the number of OpenMP threads from 1 to 23, examining the scalability of the program. 
\begin{figure}[htbp]
    \centering
    \subfigure[Total simulation time for different thread counts for direct Ewald summation for our new formulation.]{%
        \includegraphics[width=0.45\textwidth]{images/threadstimetotal.jpg}
        \label{fig:thread-a}
    }
    \hfill
    \subfigure[Total simulation time for new method with particle mesh interpolation (grid: $64 \times 64 \times 512$, order: 8) across different thread counts.]{%
        \includegraphics[width=0.45\textwidth]{images/threadstimeSPMEtotal.jpg}
        \label{fig:thread-b}
    }
    \caption{Comparison of total simulation time for new formulation across different thread counts for (a) direct Ewald summation and (b) particle mesh interpolation. Performance improves significantly with multithreading, particularly up to the number of physical cores available.}
    \label{fig:threading}
\end{figure}

The results demonstrate a significant reduction in the execution times with increasing threads, up to 12, which corresponds to the number of physical CPU cores available on the system. Beyond this point, the performance improvement begins to plateau. This behaviour shows the potential of parallelization within the bounds of the available hardware, and further improvements may require more advanced hardware.
\chapter{Conclusions and future work}
\label{Chapter6} % Change X to a consecutive number; for referencing this chapter elsewhere, use \ref{ChapterX}
\lhead{Chapter 6. \emph{Conclusions and future work}}

\section{Conclusions}

Our algorithm is straight-forward, easy to understand and integrate in the existing computer codes

\section{Scope for future work}
forces
% % Chapter Template

\chapter{Title of Chapter Seven}

\label{Chapter7} % Change X to a consecutive number; for referencing this chapter elsewhere, use \ref{ChapterX}

\lhead{Chapter 7. \emph{Title of Chapter Seven}}

\section{Background}

\lipsum[1]

\section{Results}

\subsection{Line profile analysis of XRD}
\lipsum[1]

\section{Summary}
In the present work, large strain deformation was given via plane strain compression understand the deformation behavior of the Ni-24W-22Fe alloy. The most important findings of the present work are summarized below.
\begin{enumerate}
    \item \lipsum[1]    
\end{enumerate}
% % Chapter Template

\chapter{Conclusions and future work}

\label{Chapter8} % Change X to a consecutive number; for referencing this chapter elsewhere, use \ref{ChapterX}

\lhead{Chapter 8. \emph{Conclusions and future work}}

\section{Conclusions}

\lipsum[1]

\textbf{Topic 1}

\begin{enumerate}
       
    \item \lipsum[1]

\end{enumerate}

\textbf{Topic 2}

\begin{enumerate}

    \item \lipsum[1]
     
\end{enumerate}

\section{Scope for future work}

\lipsum[1]. Some of these issues are listed:

\begin{enumerate}

    \item \lipsum[1]

\end{enumerate}


%----------------------------------------------------------------------------------------
%	THESIS CONTENT - APPENDICES
%----------------------------------------------------------------------------------------
\addtocontents{toc}{\vspace{2em}} % Add a gap in the Contents, for aesthetics
\appendix % Cue to tell LaTeX that the following 'chapters' are Appendices

% Include the appendices of the thesis as separate files from the Appendices folder
% Uncomment the lines as you write the Appendices

% Appendix Template

\chapter{Local plastic strain measurements} % Main appendix title

\label{AppendixX} % Change X to a consecutive letter; for referencing this appendix elsewhere, use \ref{AppendixX}

\lhead{Appendix A. \emph{Local plastic strain measurements}} % Change X to a consecutive letter; this is for the header on each page - perhaps a shortened title


The plastic deformation in metallic materials is of a heterogeneous nature, with strain distributed non-uniformly at the inter and intragranular levels. In order to evaluate the capabilities of full-field models for predicting the deformation arrangement, experimental techniques are required to measure the spatially resolved deformation formation in microstructures.

\section{Electron Backscatter Diffraction}
\label{Electron Backscatter Diffraction}

Electron Backscatter Diffraction (EBSD) is a technique for measuring crystal orientation using a scanning electron microscope (SEM). A focussed electron beam illuminates a spot on the surface of the sample that is inclined by $70^{\circ}$  to the beam direction. Part of this primary beam is elastically scattered from the crystal lattice. These backscattered electrons diffract through the crystal lattice as they escape the material, producing a spatial electron intensity pattern that indicates the crystal structure and orientation. The diffracted electors emanate spherically from the illuminated point on the sample surface. Since the sample is tilted, a detector can be positioned to collect a portion of the diffraction pattern on a flat plane. The collected Kikuchi pattern (named for a pioneer in electron diffraction) can then be analysed to assess the orientation of the crystal by measuring the position and orientation of bands of constructive interference \cite{maitland2007electron}.

\noindent A crystal orientation map can be collected by rastering the beam across the sample surface, collecting and analysing a Kikuchi pattern at each point. These maps contain information about grain size, grain boundary character, sample texture and alloy phase. EBSD can also be used to assess local deformations \cite{wright2011review}. This is achieved either by direct measurement of elastic strains through cross-correlation of individual Kikuchi patterns \cite{wilkinson2006high,wilkinson2006high,britton2012high} or by quantifying small changes and orientation gradients that are generated by plastic deformation \cite{kamaya2004measurement,kamaya2007local,githinji2013ebsd}.

\subsection{Measures of misorientation}
\label{Measures of misorientation}

Misorientation is the difference between two orientations and can be expressed as the minimum angle of rotation around any axis that transforms one orientation into another. The plastic deformation of a crystal due to dislocation slip induces a rotation of the crystal lattice. In an unrestrained single crystal, this leads to a net rotation of the material. However, in order to ensure that the grains remain compatible with their neighbours in a polycrystal, deformation gradients and differences in active slip system within single grains are required \cite{barbe2001intergranular}. This leads to orientation gradients that develop within the grains. Each active slip system induces rotation around a different axis, and the amount of rotation is proportional to the amount of slip. The changes in orientation are accommodated in the crystal lattice by means of geometrically necessary dislocations (GNDs), which cause a net rotation of the lattice. A GND density can be calculated from the rotation gradients of the lattice \cite{pantleon2008resolving}.

\noindent The measurement parameters for misorientation define between which orientations the misorientation is calculated and how this relates to the crystal rotation. In a class of misorientation parameters, a misorientation between neighbouring spatial points of an EBSD is calculated. This could be a single point in a single direction or, more generally, in a kernel of positions surrounding each point. The average of the misorientation at each point in the kernel results in the kernel-averaged misorientation (KAM) \cite{wright2011review}. The KAM quantifies local fluctuations in orientation. Large values indicate a large local orientation gradient resulting from a change in the active slip system or an abrupt change in slip activity. Examples of KAM maps taken from a plastically deformed sample are shown in figure \ref{fig:Measures of misorientation1}a-b. The EBSD maps are taken at two different EBSD step sizes and show that KAM values are sensitive to the chosen EBSD step size.

\begin{figure}
    \centering
    \includegraphics[scale=1.5]{Pictures/KAM.jpg}
    \caption{Examples of misorientation parameters for the same region of material. a) and b) are KAM calculated for a 3x3 pixel $^{2}$ kernel for EBSD step sizes of $0.2 \mu \mathrm{m}$ and $0.4 \mu \mathrm{m} .$ c) is ROD calculated to each grains average orientation and d) ROD calculated to the point in each grain with minimum KAM. Taken from \cite{wright2011review}.}
    \label{fig:Measures of misorientation1}
\end{figure}

KAM does not indicate gradual changes of orientation relating to gradients of deformation. This can be measured using a second class of parameters, the reference orientation deviation (ROD) \cite{wright2011review}. As the name suggests ROD is the misorientation between each point of a map to a reference orientation.A reference orientation must be defined separately for each grain in the EBSD map, since the misorientation between two grains is generally greater than any gradient in it. Common choices for the reference orientation are the mean orientation of each grain or the point in each grain with the minimum KAM value. Figure \ref{fig:Measures of misorientation1}c-d shows a ROD map for each of these reference orientation choices. Similar gradients can be seen in each map but with different magnitudes of the ROD values, whereby the choice of the reference orientation only changes the zero point of the misorientation. Since ROD measures misorientation relative to a fixed reference, it is not sensitive to step size changes.

\noindent The use of misorientation can indicate intense strain in the grains of a microstructure \cite{kamaya2006quantification}, but cannot identify magnitudes of strains. The distribution of the intragranular plastic strain is also not well predicted due to misorientation compared to local strain measurements from image correlation techniques \cite{kamaya2007local}. The misorientation, however, provides a good measure of the microstructure evolution induced by plastic deformation.
%\chapter{C/C++ Programs}
\label{AppendixB} % use \ref{AppendixB}
\lhead{Appendix B. \emph{C/C++ Programs}}

\section{2D Ewald}
    \subsection{Main file}
    \lstinputlisting[language=C]{CodeFiles/Ewald2D/main.cpp}
    
    \subsection{dist function}
    \lstinputlisting[language=C]{CodeFiles/Ewald2D/dist.cpp}
    
    \subsection{Real Space Energy function}
    \lstinputlisting[language=C]{CodeFiles/Ewald2D/real.cpp}
    
    \subsection{Self Energy function}
    \lstinputlisting[language=C]{CodeFiles/Ewald2D/self.cpp}
    
    \subsection{2D-EW}
    \textbf{Reciprocal Space Energy (k $=$ 0)}
    \lstinputlisting[language=C]{CodeFiles/Ewald2D/reci0.cpp}
    
    \textbf{Reciprocal Space Energy (k $\neq$ 0)}
    \lstinputlisting[language=C]{CodeFiles/Ewald2D/reci_integral.cpp}
    
    \subsection{2D-PME}
    \lstinputlisting[language=C]{CodeFiles/Ewald2D/reci_fft.cpp}
    
    \subsection{Helper functions}
    \lstinputlisting[language=C]{CodeFiles/Ewald2D/func.cpp}
    
    \subsection{New 2D Ewald Method}
        \subsubsection{Direct Ewald}
        \lstinputlisting[language=C]{CodeFiles/Ewald2D/New2DEwald.cpp}
        
        \subsubsection{Particle Mesh Adaptation}
        \lstinputlisting[language=C]{CodeFiles/Ewald2D/PM2DEwald.cpp}

        \subsubsection{Screening Factor Array}
        \lstinputlisting[language=C]{CodeFiles/Ewald2D/screen.cpp}
        
% \section{New Method for 2D Ewald}

%\input{Appendices/AppendixC}

\addtocontents{toc}{\vspace{2em}} % Add a gap in the Contents, for aesthetics

\backmatter

%----------------------------------------------------------------------------------------
%	BIBLIOGRAPHY
%----------------------------------------------------------------------------------------
\nocite{*}
\label{Bibliography}

\lhead{\emph{Bibliography}} % Change the page header to say "Bibliography"
\bibliographystyle{ieeetr} % Use the "custom" BibTeX style for formatting the Bibliography
\bibliography{Bibliography} % The references (bibliography) information are stored in the file named "Bibliography.bib"

\end{document}  