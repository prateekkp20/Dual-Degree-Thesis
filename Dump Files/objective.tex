Extensive research has been done to reduce the computational expense of long-range interactions in molecular simulations. Though the computer hardware is seeing continuous advances, a critical need remains for more efficient and scalable techniques that can accurately model and simulate large physical systems. Since the computational effort required to compute long-range coulombic interactions scales quadratically with the number of ions, developing algorithms that significantly reduce these costs, particularly for systems that exhibit reduced periodicities, is essential. 

The conventional approach for modelling slab geometries introduces a dipole correction \cite{dipole-yeh-berkowitz} to the Coulomb interactions within a fully periodic framework and introduces an optimal vacuum in the non-periodic direction to minimize residual errors. Alternatively, exact formulations of the Ewald summation have been developed for systems with two-dimensional periodicity \cite{kawata,PARRY1975433,de1979electrostatic,Heyes19771485}, accounting rigorously for the anisotropic boundary conditions inherent to slab configurations. These methods accurately capture long-range electrostatics without relying on artificial corrections; however, their computational cost typically exceeds $O(NlogN)$, rendering them impractical for large-scale simulations.

The primary objective of this thesis is to develop and implement an exact 2D Ewald summation algorithm that achieves the same computational complexity as the traditional 3D Ewald method, while offering improved efficiency and scalability. Additionally, this work focuses on writing high-performance, optimized code, emphasizing efficient memory access, vectorization, and cache-friendly data structures, and further optimizing the computational performance through parallelization using OpenMP. Through the development of novel algorithm tailored for 2D periodic systems, combined with low-level code optimizations facilitated by performance profiling using Intel\textsuperscript{\textregistered} VTune and parallelization using OpenMP, this thesis aims to provide a robust and practical framework for large-scale \acs{MD} simulations that can be useful to study some fundamental physical and chemical phenomena like thin films, surface science,  electrochemical interfaces and adsorption.