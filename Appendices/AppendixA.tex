\chapter{Real to Fourier Space Transformation using Poisson Summation}
\label{AppendixA} 
\lhead{Appendix A. \emph{Real to Fourier Space Transformation using Poisson Summation}} 
Let us consider the following periodic functions defined along different spatial directions.
In the $x$-direction, we define the function $F(x)$ as:
\begin{flalign*}
    F(x) &= \sum_{{m_x}=-\infty}^{\infty}exp\left[-(x+m_xL_x)^2 t^2\right]
\end{flalign*}
and similarly, in the $z$-direction, the function $F(z)$ is given by:
\begin{flalign*}
    F(z) &= \sum_{{m_z}=-\infty}^{\infty}exp\left[-(z+m_zL_z)^2 t^2\right]\phi(z+m_zL_z) 
    % \\ &= \sum_{{m_z}=-\infty}^{\infty}exp\left[-(z+m_zL_z)^2 t^2\right]\left[\frac{1}{1+ exp(-\gamma(L_z(0.5+m_z)+z))} + \frac{1}{1+ exp(-\gamma(L_z(0.5-m_z)-z))} -1\right]
\end{flalign*}
where $L_x$ and $L_z$ denote the periodic lengths in the respective directions, $t$ is a constant, and $\phi(z)$ is a top-hat function.
Note that both functions $F(x)$ and $F(z)$ are periodic due to the infinite summation. Specifically,\[
F(x + L_x) = F(x) \quad \text{and} \quad F(z + L_z) = F(z).
\]
According to the Poisson summation formula, any periodic function with period \( L_z \) can be represented as a Fourier series. Hence, \( F(z) \) can be written as:
\begin{flalign*}
    F(z) &= \sum_{k=-\infty}^{\infty} C_k \hspace{1mm}exp(i\frac{2\pi k z}{L_z}) 
\end{flalign*}
where $C_k$ are constants. To obtain the values of $C_k$, we multiply each side by $exp(-i\frac{2\pi n z}{L_z})$ and integrate in z from 0 to $L_z$.
\begin{flalign*}
    \int_{0}^{L_z}C_{k_z}(t)dz &= \sum_{{m_z}=-\infty}^{\infty} \int_{0}^{L_z} dz\hspace{1mm}exp(-i\frac{2\pi n z}{L_z}) \hspace{1mm} exp\left[-(z+m_zL_z)^2 t^2\right] \, \phi(z+m_zL_z)
    % \\ &=\sum_{{m_z}=-\infty}^{\infty} \int_{0}^{L_z} dz\hspace{1mm}exp(-i\frac{2\pi n z}{L_z}) \hspace{1mm} exp\left[-(z+m_zL_z)^2 t^2\right]\times \\&\hspace{40mm}\left[\frac{1}{1+ exp(-\gamma(L_z(0.5+m_z)+z))} + \frac{1}{1+ exp(-\gamma(L_z(0.5-m_z)-z))} -1\right]
\end{flalign*}
Now substituting $s = z + mL_z$, and realizing that $exp(-i2\pi m n)=1$, we obtain
\begin{flalign}
   \nonumber C_{k_z}(t) &= \frac{1}{L_z}\sum_{{m_z}=-\infty}^{\infty} \int_{mL_z}^{(m+1)L_z} ds\, exp(-s^2t^2)\,exp(-i\frac{2\pi n s}{L_z}) \, \phi(s)
    \\ &=\frac{1}{L_z}\int_{-\infty}^{\infty}ds\, exp(-s^2t^2)\,exp(-i\frac{2\pi n s}{L_z}) \, \phi(s) \label{eq:ck}
\end{flalign}
Due to the complex form of $\phi(s)$ we cannot obtain a closed-form expression for the above integral and will have to solve using numerical methods.
% \colorbox{yellow}{no closed form integral expression} \\
To get the coefficients for the X and Y direction function, put $\phi(s) = 1$, and we can obtain the following closed-form expression for $C_{k_x}$ and $C_{k_y}$:
\begin{flalign*}
   C_{k_x}(t) &= \frac{1}{L_x}\sqrt{\frac{\pi}{t}} exp\left(\frac{-\pi^2k_x^2}{L_x^2t}\right)
\end{flalign*}
So our final expression to transform from real to reciprocal space would be:
\begin{flalign*}
    F(x) &=\sum_{{m_x}=-\infty}^{\infty}exp\left[-(x+m_xL_x)^2 t^2\right] = \frac{1}{L_x}\sqrt{\frac{\pi}{t}}\sum_{k_x=-\infty}^{\infty}exp\left(\frac{-\pi^2k^2}{L_x^2t}+i\frac{2\pi k_x x}{L_x}\right) 
    % \\ \sum_{{m_z}=-\infty}^{\infty}exp\left[-(z+m_zL_z)^2 t^2\right]\phi(z+m_zL_z) &=  \frac{1}{L_z}\sum_{k=-\infty}^{\infty}  \left[ \int_{-\infty}^{\infty}ds\, exp(-s^2t^2)\,exp(-i\frac{2\pi n s}{L_z}) \, \phi(s)\right]  exp\left(i\frac{2\pi kx}{L_x}\right)  
    \\ F(z) &=\sum_{{m_z}=-\infty}^{\infty}exp\left[-(z+m_zL_z)^2 t^2\right]\phi(z+m_zL_z) =  \sum_{k_z=-\infty}^{\infty}  C_{k_z}(t)  \, exp\left(i\frac{2\pi k_z z}{L_z}\right)  
\end{flalign*}
The value of $C_{k_z}(t)$ comes from equation \ref{eq:ck}