\chapter{Numerical Evaluation: Accuracy and Efficiency}

\label{Chapter7} % Change X to a consecutive number; for referencing this chapter elsewhere, use \ref{ChapterX}

\lhead{Chapter 7. \emph{Numerical Evaluation: Accuracy and Efficiency}}
Simulations run on 12\textsuperscript{th} Gen Intel\textsuperscript{\textregistered} Core\texttrademark{} i5-12500~$\times$~12 core processor
\section{Convergence}
We took a system of 10,000 Na and Cl ions in a 25 \AA box for our benchmarking. Based on our analysis for various gamma values, we calculated the reciprocal space energies for a given system and checked for the convergence of the sum. 
% \begin{figure}[htbp]
\begin{figure}[]
    \centering
    \includegraphics[scale=0.4]{images/logerror_vs_kz_forreport.jpg}
    \caption{Convergence of relative errors in $U_{LR}$ with $k_z$ for various values of $\gamma$.}
    \label{fig:result1}
\end{figure}
\section{Particle Mesh Ewald-Modified}
\section{Performance Scaling}
\subsection{Size Increase}
\subsection{Multi-threading}

\section{Summary}