% Chapter Template

\chapter{Development of a New Algorithm / Our Work}

\label{Chapter4}

\lhead{Chapter 4. \emph{Development of a New Algorithm / Our Work}} 
In the case of a two-dimensional periodic system, the summation over the $m_z$ component in Eq.~(\ref{eq:coul}) is omitted, which alters the Ewald formulation relative to the fully periodic (3D) case and leads to a Fourier integral, thereby increasing the computational cost. To address this difference, the $m_z$ summation is retained as in the three-dimensional periodic system, and a top-hat function is introduced. This function serves to eliminate contributions from periodic images along the $z$-direction, effectively treating these images as non-existent.

A vacuum is introduced within the unit cell along the $z$-axis, resulting in a total side length of $ L_z$. This length must satisfy the condition $L_z \geq 2L$, where $L$ is the side length of the original simulation box. This condition ensures that the vacuum region is large enough to prevent atoms from adjacent periodic images from interacting with the original cell. Such interactions can lead to artificial periodic effects that would compromise the accuracy of the simulation.
\section{Top-hat function}
\begin{figure}[]% if you want to fix the position of this image, add H inside the []
    \centering
    \includegraphics[width=\linewidth]{images/simulationcell.jpg}
    \caption{insert some caption}
    \label{fig:simulation_cell}
\end{figure}

To exclude the system's interactions along the $z$-direction, a spatially dependent window function $\phi(z)$ is introduced. The simplest version of $\phi(z)$ is a top-hat function defined as:
\begin{flalign*} 
    \phi(z) = 
    \begin{cases} 
        1 & \text{if } z \in (-\frac{L_z}{2}, \frac{L_z}{2}) \\
        0 & \text{otherwise}
    \end{cases}
\end{flalign*}

While this function is straightforward, its sharp discontinuities at the edges can introduce numerical artefacts, especially in methods sensitive to derivatives or smoothness (e.g., force field calculations). To address this, the discontinuous top-hat function is replaced with a smooth approximation that retains the essential features of the original function while providing a differentiable and numerically stable function.
The smoothed version of $\phi(z)$ is expressed as:
\begin{flalign}
    \phi(z) &= \frac{1}{1+ exp(-\gamma(0.5L_z+z))} + \frac{1}{1+ exp(-\gamma(0.5L_z-z))} -1 \label{eq:1}
\end{flalign}

This function uses a linear combination of two sigmoid functions to make a symmetrical top-hat function about the origin. The positive constant \textbf{$\gamma$} determines the steepness of the top-hat around the edges. The behaviour of $\phi(z)$ Eq. (\ref{eq:1}) is shown in Fig. (\ref{fig:tophat}).
Other forms of the top-hat functions can also be used; some examples are as follows:
\begin{flalign}
        \phi(z) &= \frac{1}{2}[ \tanh(\gamma(x + 0.5L)) -\tanh(\gamma(x - 0.5L)) ] \\
        \phi(z) &= \frac{2}{\pi}\left[ \arctan \left( e^{\gamma (-x + 0.5L)} \right) + \arctan \left( e^{\gamma (x + 0.5L)} \right) \right]- 1 
\end{flalign}
\begin{figure}[htbp]
    \centering
    \includegraphics[width=0.7\linewidth]{images/TopHat2.png}
    \caption{Visualisation of the top-hat function for different $\gamma$. As $\gamma$ increases, the function transitions more sharply at the boundaries, approaching an ideal window function.}
    % \parbox{0.5\linewidth}{\justifying \caption{Visualization of top-hat function for different $\gamma$. As $\gamma$ increases, the function transitions more sharply at the boundaries, approaching an ideal window function.}}
    \label{fig:tophat}
\end{figure}
\section{Incorporating the top-hat function}
To evaluate the long-range contribution to the electrostatic energy in a two-dimensional (2D) periodic system, the derivation begins with the standard three-dimensional (3D) Ewald summation expression. A top-hat function, denoted by $\phi(z)$, is introduced to eliminate the influence of periodic images along the non-periodic $z$-direction. This function ensures that only the real physical interactions in the $z$-direction are retained, effectively reducing the system to 2D while preserving the 3D summation structure.

The long-range part of the Coulomb interaction, initially expressed as $\frac{\text{erf}(\alpha r)}{r}$, is converted into an integral form using the identity:
$$
\frac{\text{erf}(\alpha r)}{r} = \frac{2}{\sqrt{\pi}} \int_0^\alpha e^{-r^2 t^2} dt.
$$
This transformation separates the exponential term into components along the $x$, $y$, and $z$ directions.
\begin{flalign}
    \nonumber(4\pi\epsilon_o)U^{LR} & = \frac{1}{2}\sum_{\vec{M}=-\infty}^{\infty}\sum_{i=1}^{n_p}\sum_{j=1}^{n_p}q_i q_j \frac{{erf}(\alpha|\vec{r_i}-\vec{r_j}+\vec{M}.\vec{L}|)}{|\vec{r_i}-\vec{r_j}+\vec{M}.\vec{L}|} \,\phi(z_{ij}+m_zL_z)\quad\quad\quad\quad
\end{flalign}
\begin{flalign}
    \nonumber\quad\quad\quad\quad\quad& = \frac{1}{2}\sum_{\vec{M}=-\infty}^{\infty}\sum_{i=1}^{n_p}\sum_{j=1}^{n_p}\frac{2q_i q_j}{\sqrt{\pi}}\int_{0}^{\alpha}dt\,  {exp}(-|\vec{r_i}-\vec{r_j}+\vec{M}.\vec{L}|^2 t^2)\,\phi(z_{ij}+m_zL_z)\\
    \nonumber &=\frac{1}{2}\sum_{\vec{M}=-\infty}^{\infty}\sum_{i=1}^{n_p}\sum_{j=1}^{n_p}\frac{2q_i q_j}{\sqrt{\pi}}\int_{0}^{\alpha}dt\,  {exp}\left[-(x_{ij}+m_xL_x)^2 t^2\right] \\
    &\quad\quad\quad\times{exp}\left[-(y_{ij}+m_yL_y)^2 t^2\right]\times{exp}\left[-(z_{ij}+m_zL_z)^2 t^2\right] \, \phi(z_{ij}+m_zL_z)
\end{flalign}
The Poisson summation formula is used to convert the expression from real space to reciprocal space. The detailed derivation of this transformation is provided in the \colorbox{yellow}{Appendix}.
\begin{flalign}
    \nonumber(4\pi\epsilon_o)U^{LR}& =\frac{\sqrt{\pi}}{L_xL_y}\sum_{\vec{\mathbf{k}}=-\infty}^{\infty}\sum_{j=1}^{n_p}\sum_{k=1}^{n_p}q_j q_k \int_{0}^{\alpha}\frac{dt}{t^2}\,{exp}\left(\frac{-\pi^2k_x^2}{L_x^2t^2}+i \frac{2\pi k_xx_{ij}}{L_x}\right)
    \\&\quad\quad\quad
    \times\,{exp}\left(\frac{-\pi^2k_y^2}{L_y^2t^2}+i\frac{2\pi k_yy_{ij}}{L_y}\right)\times C_{k_z}(t)\,{ exp}\left(i\frac{2\pi k_zz_{ij}}{L_z}\right)\label{eq:transform}
\end{flalign}
The function $C_{k_z}(t)$ is given by
\begin{flalign}
     C_{k_z}(t) &=\frac{1}{L_z}\int_{-\infty}^{\infty}ds\,exp(-i\frac{2\pi n s}{L_z})\, exp(-s^2t^2)\, \phi(s) \label{eq:Cz}
     % \\ &=\frac{1}{L_z}\int_{-\infty}^{\infty}ds\hspace{1mm}exp(-i\frac{2\pi n s}{L_z})\hspace{1mm} exp(-s^2t^2)\left[\frac{1}{1+ exp(-\gamma(0.5L_z+s))} + \frac{1}{1+ exp(-\gamma(0.5L_z-s))} -1\right]
\end{flalign}
Eq. (\ref{eq:transform}) is written using the reciprocal vector formulation as
\begin{flalign*}
    \nonumber(4\pi\epsilon_o)U^{LR}& =\frac{\sqrt{\pi}}{L_xL_y}\sum_{\vec{\mathbf{k}}=-\infty}^{\infty}\sum_{i=1}^{n_p}\sum_{j=1}^{n_p}q_i q_j\int_{0}^{\alpha}\frac{dt}{t^2}\,{ exp}\left[i\vec G.(\vec r_i-\vec r_j)\right]\times C_{k_z}(t)\,{ exp}\left(\frac{-1}{4t^2}|\vec G|_{xy}^2\right)\\
    &=\frac{\sqrt{\pi}}{L_xL_y}\sum_{\vec{\mathbf{k}}=-\infty}^{\infty}\left[ \int_{0}^{\alpha}\frac{dt}{t^2}C_{k_z}(t)\,{exp}\left(\frac{-1}{4t^2}|\vec G|_{xy}^2\right)\right]\sum_{j=1}^{n_p}\sum_{k=1}^{n_p}q_i q_j\,{ exp}\left[i\vec G.(\vec r_i-\vec r_j)\right]
\end{flalign*}
The double summation over indices $j$ and $i$ is reformulated as a single summation incorporating the structure factor Eq. (\ref{eq:structurefactor}).
Hence, the final expression for the two-dimensional Ewald summation takes the form
\begin{flalign}
    (4\pi\epsilon_o)U^{LR}& =\frac{\sqrt{\pi}}{L_xL_y}\sum_{\vec{{k}}=-\infty}^{\infty}{}^\prime\left[ \int_{0}^{\alpha}\frac{dt}{t^2}C_{k_z}(t){exp}\left(\frac{-1}{4t^2}|\vec G|_{xy}^2\right)\right] |\,S(\vec G)\,|^2
\end{flalign}
The prime symbol (${}^\prime$) indicates that the $\vec{k} = 0$ component is omitted from the summation, since it vanishes when the system is electrically neutral, that is, when the total charge sums to zero.
Define $\Lambda(p_1,p_2,p_3)$ as
\begin{flalign}
    \nonumber \Lambda(p_x,p_y,p_z) &= \frac{\sqrt{\pi}}{L_xL_y}\sum_{\vec{\mathbf{k}}=-\infty}^{\infty}{}^\prime \left[ \int_{0}^{\alpha}\frac{dt}{t^2}C_{k_z}(t){exp}\left(\frac{-1}{4t^2}|\vec G|_{xy}^2\right)\right] 
    \\&\times B_{3D}(k_x,k_y,k_z)\times \exp\left( i \frac{2\pi k_x p_x}{K_x} + i \frac{2\pi k_y p_y}{K_y} + i \frac{2\pi k_z p_z}{K_z} \right)
    \\&=\mathcal{F}_{3D}(E_{2D-M}.B_{3D})(p_x,p_y,p_z)
\end{flalign}
where the new screening factor array $E_{2D-M}$ is defined as
\begin{flalign}
    E_{2D-M}(k_x,k_y,k_z) &= \frac{\sqrt{\pi}}{L_xL_y}\left[ \int_{0}^{\alpha}\frac{dt}{t^2}C_{k_z}(t){exp}\left(\frac{-1}{4t^2}|\vec G|_{xy}^2\right)\right] 
\end{flalign}
Note that $E_{2D-M}\cdot B_{3D}(k_x,k_y,k_z) = \mathcal{F}^{-1}_{3D}(\Lambda)$. 
Furthermore, the structure factor can be efficiently calculated using the Smooth Particle Mesh Ewald (SPME) method by using Fast Fourier Transforms (FFTs).
\begin{flalign}
    \nonumber (4\pi\epsilon_o)U^{LR}  & \approx \frac{\sqrt{\pi}}{L_xL_y}\sum_{\vec{\mathbf{k}}=-\infty}^{\infty}{}^{\prime}\left[ \int_{0}^{\alpha}\frac{dt}{t^2}C_{k_z}(t){exp}\left(\frac{-1}{4t^2}|\vec G|_{xy}^2\right)\right]\,B_{3D}(k_x,k_y,k_z)
    \\& \quad\quad\quad\quad\quad\quad\quad\quad\times \left|\mathcal{F}_{3D}(Q)(k_x,k_y,k_z)\right|^2 \label{eq:newreci2DSPME}\\
    \nonumber&= \sum_{k_x=0}^{K_1-1} \sum_{k_y=0}^{K_2-1} \sum_{k_z=0}^{K_3-1}\mathcal{F}^{-1}_{3D}(\Lambda)(k_x,k_y,k_z)\cdot \cdot \mathcal{F}_{3D}(Q)(k_x,k_y,k_z) \\
    \nonumber &\quad\quad\quad\quad\quad\cdot K_xK_yK_z\,\mathcal{F}^{-1}_{3D}(Q)(k_x,k_y,k_z) \\
    &=\sum_{k_x=0}^{K_1-1} \sum_{k_y=0}^{K_2-1} \sum_{k_z=0}^{K_3-1} Q(k_x,k_y,k_z) \cdot (\Lambda \star Q)(k_x,k_y,k_z)
\end{flalign}
The reciprocal space force is obtained as
\begin{flalign}
    \frac{\partial U^{LR}}{\partial r_{\beta i}} &= 2\sum_{k_x=0}^{K_1-1} \sum_{k_y=0}^{K_2-1} \sum_{k_z=0}^{K_3-1} \frac{\partial Q(k_x,k_y,k_z)}{\partial r_{\beta i}} \cdot (\Lambda \star Q)(k_x,k_y,k_z)
\end{flalign}
\section{Choosing Parameters}
\subsection{Selection of $\gamma$}
The Fourier transform of a top-hat function is known to yield oscillatory output. 
This section seeks to identify suitable values of $\gamma$ that can reduce these oscillations.

In particular, it is noted that the expression in Eq.~(\ref{eq:Cz}) represents the Fourier transform of the product of a top-hat function, $\phi(s)$, and a Gaussian function. It is hypothesised that most of the oscillations in the calculations would arise from the top-hat component. To investigate this, the behaviour of its Fourier transform is examined by varying the parameters $\gamma$ and $L_z$, to determine configurations that minimise the undesirable oscillatory effects.

In Fig.~\ref{fig:tophat}, as the parameter $\gamma$ decreases, the top-hat function becomes increasingly localised and narrower in real space. Conversely, larger values of $\gamma$ produce a broader profile. However, as shown in Fig.~\ref{fig:fourieroftophatvarygammaL300}, broader top-hat functions lead to extended and persistent oscillations in the frequency domain, which can adversely affect the convergence of numerical computations for reciprocal space energies. To mitigate these oscillations, smaller values of $\gamma$ are preferred, resulting in narrower real-space functions with more rapidly decaying Fourier transforms. The trade-off, however, is that a smaller $\gamma$ reduces the region over which the top-hat function maintains a value of exactly one. To accommodate this, the domain length $L_z$ must be increased to ensure the simulation box remains entirely within this central region.

\begin{figure}[htbp]
  \centering
  \includegraphics[width=\linewidth]{images/fourieroftophatvarygammaL300.jpg}
  \caption{Fourier Transform of Top-hat Function with Varying $\gamma$, $L_z = 300$}
  \label{fig:fourieroftophatvarygammaL300}
\end{figure}
  
  % \vspace{1em} % Optional spacing between figures

\begin{figure}[htbp]
  \centering
  \includegraphics[width=\linewidth]{images/fourieroftophatvaryL_gamma0.5.jpg}
  \caption{Fourier Transform of Top-hat Function with Varying $L_z$, $\gamma = 0.5$}
  \label{fig:fourieroftophatvaryL_gamma0}
\end{figure}

Fig.~\ref{fig:fourieroftophatvaryL_gamma0} supports this choice by showing the Fourier Transform for varying $L_z$ at a fixed $\gamma = 0.5$. It demonstrates that changing $L_z$ does not impact the convergence or magnitude of the oscillations in the frequency domain. Although the damping frequency, where oscillations begin to decay, shifts higher with larger $L_z$values, this does not affect the essential behaviour of the transform. This confirms that increasing $L_z$ when using smaller $\gamma$ values does not introduce computational issues, making this a suitable strategy for reducing oscillations while preserving accuracy.

\subsection{Selection of  $L_z$}
To ensure that the simulation cell is completely contained within the region where $\phi(z) = 1$, the smallest possible value for $L_z$ is selected. This approach guarantees the proper placement of the simulation cell and reduces the number of grid points required for the Smooth Particle Mesh Ewald (SPME) method, thereby enhancing computational efficiency. To ascertain the optimal value of $L_z$, a binary search-based algorithm (refer to \textbf{Alg. \ref{alg:vacuum}}) was implemented to determine the box length.

\input{binaryAlgorithm}