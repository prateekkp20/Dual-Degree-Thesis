% Chapter Template

\chapter{Performance Analysis and Optimisation}

\label{Chapter6}

\lhead{Chapter 6. \emph{Performance Analysis and Optimisation}}

\section{Memory Optimisation via Array Flattening}
In the implementation, data involving multiple dimensions must be stored in several parts of the program, such as atom positions and charge spreading array for the SPME. While multidimensional arrays are a natural choice for such data, they have drawbacks. 

Dynamically allocated multidimensional arrays often lead to scattered memory layouts and multiple pointer dereferences. This results in poor cache performance and added complexity in memory management. 
To address this, a one-dimensional array was used to represent the multidimensional structure. For an array with rank $d$ and dimensions $n_1\times \ldots \times n_d$, an element at ($i_1\times \ldots \times i_d$) maps to:
\begin{flalign*}
    i_d + n_d \cdot \left( i_{d-1} + n_{d-1} \cdot \left( \ldots + n_2 \cdot i_1 \right)\right)
\end{flalign*}
This approach reduced overhead, improved memory locality, and allowed faster access through direct indexing.

% the indexing line is taken from the fftw tutorial website.
% https://www.fftw.org/fftw2_doc/fftw_2.html


