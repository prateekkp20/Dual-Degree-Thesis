% Chapter Template

\chapter{Coulombic Interaction and Ewald Summation} % Main chapter title

\label{Chapter3} % Change X to a consecutive number; for referencing this chapter elsewhere, use \ref{ChapterX}

\lhead{Chapter 3. \emph{Coulombic Interaction and Ewald Summation}} 

For a charge neutral system ($\sum_i q_i=0$) with $n_p$ charges, $q_1,q_2....,q_{n_{p}}$ at $\vec{r_1},\vec{r_2},...,\vec{r_{n_p}}$, the Coulomb interaction energy is given by
\begin{flalign}
    (4\pi\epsilon_o)U &= \frac{1}{2}\sum_{\vec{M}= -\infty}^{\infty}{' \sum_{i=1}^{n_p}\sum_{j=1}^{n_p} \frac{q_iq_j}{|\vec{r_i}-\vec{r_j}+\vec{M}.\vec{L}|}}\label{eq:coul}
\end{flalign}
where the prime (${}^\prime$) symbol is introduced to exclude the term $j = i$, when $\vec{M}=0 $. This summation is conditionally convergent, meaning that the final value depends on the specific order in which the terms are summed. Given that $\rho(\vec{r})$ is the charge density at any point $\vec{r}$ and $\phi(\vec{r})$ is the potential field at point $\vec{r}$. This interaction can also be formulated as
\begin{flalign}
     (4\pi\epsilon_o)U &= \frac{1}{2} \int_{-\infty}^{\infty} d\vec{r_2}\, \rho(\vec{r_2}) \int_{-\infty}^{\infty} \frac{d\vec{r_1}\, \rho(\vec{r_1})}{|\vec{r_1}-\vec{r_2}|} \\
     &= \frac{1}{2}\int_{-\infty}^{\infty} d\vec{r_2}\, \rho(\vec r_2)\phi(\vec r_2)
\end{flalign}
For a point charge system, $\rho(\vec{r})$ is given by
\begin{equation}
    \rho(\vec{r})=\sum_{\vec{M}=-\infty}^{\infty}\sum_{j=1}^{n_p}q_j\delta(\vec{r}-\vec{r_j}+\vec{M}.\vec{L})
\end{equation}
Here $\vec{M} \in \mathbb{Z}^n$ for an n-dimensional periodic system. To simulate ionic systems, it is essential to compute these interactions acting on each ion. However, the computational cost of these calculations typically scales quadratically with the number of ions, i.e., $O(N^2)$. This becomes computationally prohibitive for realistic molecular simulations involving as many as $10^5$ ions with PBCs.

\section{Ewald Summation Method}
The Ewald summation method, developed by Ewald in 1921, is a classical approach used to compute long-range Coulombic interactions in periodic systems. Ewald proposed decomposing the $1/r$ potential into two rapidly converging parts
\begin{flalign*}
\frac{1}{r} = \frac{{erfc}(\alpha r)}{r} + \frac{{erf}(\alpha r)}{r}
\end{flalign*}
Here, $\alpha$ is the Ewald splitting parameter that controls the convergence of the two parts. The first term represents a short-range interaction computed in real space, while the second term corresponds to a long-range interaction evaluated in reciprocal (Fourier) space. Additionally, a self-interaction correction is applied to account for the spurious interaction of each charge with its smeared-out image.

The total electrostatic energy is thus expressed as
\begin{flalign}
U = U^{SR} + U^{LR} + U^{S}
\end{flalign}
This decomposition ensures faster convergence and significantly reduces the computational cost of simulating systems with periodic boundary conditions.

\section{Three-Dimensional Periodic Systems}
Define the reciprocal vector $\vec{G}$ as $\vec G =  \frac{k_x2\pi}{L_x}\hat x+\frac{k_y2\pi}{L_y}\hat y+\frac{k_z2\pi}{L_z}\hat z$, where $k_x$, $k_y$, $k_z$ $\in$ $\mathbb{Z}^3$, and the structure factor $S(\vec G)$ as
\begin{flalign}
    S(\vec G) &= \sum_{j=1}^{n_p}q_j\,exp(i\vec G.\vec r_j)
\end{flalign}
The Coulombic interaction in Eq.~(\ref{eq:coul}) for a three-dimensional periodic system is expressed using the Ewald summation as
\begin{flalign}
    \nonumber (4\pi\epsilon_o)U &= U^{LR} + U^{S} + U^{SR}
\end{flalign}
Each of these terms is written as
\begin{flalign}
    U^{LR}& =\frac{2\pi}{L_xL_yL_z}\sum_{\mathbf{k}=-\infty}^{\infty}{}^{\prime}\frac{1}{|\vec{G}|^2}\,{exp}\left(\frac{-1}{4\alpha^2}|\vec{G}|^2\right)|\,S(\vec{G})\,|^2\, \\
    U^{S} &= -\frac{\alpha}{\sqrt{\pi}}\sum_{i=1}^{n_p} q_i^2  \\
    U^{SR}&=\frac{1}{2}\sum_{i=1}^{n_p}{}^\prime\sum_{j=1}^{n_p}q_i q_j\frac{erfc(\alpha|\vec{r_j}-\vec{r_i}|)}{|\vec{r_j}-\vec{r_i}|}
\end{flalign}
The computational cost of the traditional Ewald summation method, which scales as $O(N^2)$ and can be reduced to $O(N^\frac{3}{2})$ with optimised approaches, remains prohibitive for extensive molecular simulations. This limitation has motivated the development of more efficient techniques such as the Particle Mesh Ewald (PME) and Smooth Particle Mesh Ewald (SPME) methods. 

The electrostatic force on $i^{th}$ atom in each direction ($\beta = x,y,z$) can be obtained by taking a derivative of the above components with respect to $\vec{r_{\beta i}}$. Each of the individual term would be defined as $\partial U^{LR}/\partial  r_{\beta i}$ (reciprocal space force), $\partial U^{SR}/\partial r_{\beta i}$ (real space force) and $\partial U^{S}/\partial r_{\beta i}$ (correction force).

\subsection{Smooth Particle Mesh Ewald}
The Particle Mesh Ewald (PME) method was first introduced by Hockney and Eastwood, utilising Laplace interpolation techniques to assign charges onto a regular grid. However, the discontinuous nature of Laplace interpolations posed challenges in accurately computing forces. To address this, the Smooth Particle Mesh Ewald (SPME) method was later developed by Essmann\textit{ et al.}, which employs exponential B-splines to smoothly interpolate the charge distribution on the grid, enabling more accurate and efficient computations. These approaches employ Fast Fourier Transforms (FFT), thereby decreasing the computational complexity from $O(N^2)$ to $O(Nlog(N))$, with a multiplicative constant that varies based on the desired level of accuracy.

In the SPME method, the structure factor in the reciprocal space sum is expressed as
\begin{equation}
\exp(2\pi i \mathbf{k} \cdot \mathbf{r}) =
\exp\left(2\pi i \frac{k_x u_1}{K_1} \right)
\exp\left(2\pi i \frac{k_y u_2}{K_2} \right)
\exp\left(2\pi i \frac{k_z u_3}{K_3} \right),
\end{equation}
where $\mathbf{u} = \mathbf{K}.\mathbf{r^*}$ are the scaled fractional coordinates, where $\mathbf{K}$ is the grid points vector. To efficiently compute this on a mesh, each term is approximated using exponential B-splines
\begin{equation}
    \exp\left(2\pi i \frac{k_\lambda}{K_\lambda} u_\lambda\right) \approx 
    b \left(2\pi \frac{k_\lambda}{K_\lambda},n\right) \sum_{m=-\infty}^{\infty} M_n(u_\lambda - m) 
    \cdot \exp\left(2\pi i \frac{k_\lambda}{K_\lambda} m\right),\label{eq:bspline}
\end{equation}
and $\lambda$ is the direction ($\lambda = $ x, y, z) and $n$ as the order of B-spline interpolation, with normalisation
\begin{equation}
    b(u,v) = \frac{\exp\left(i (v - 1) u\right)}
       {\sum_{m=0}^{n-2} M_n(m+1) \exp\left(i um\right)}
\end{equation}
The B-spline functions are defined recursively. The second-order B-spline is 
$M_2(u) = 
1 - |u - 1| \text{ for } 0 \le u \le 2$, and zero otherwise, and for higher orders \( n > 2 \), the recursion is defined by Cox–de Boor formula as
\begin{equation}
M_n(u) = \frac{u}{n-1} M_{n-1}(u) + \frac{n - u}{n - 1} M_{n-1}(u - 1).
\end{equation}
The structure factor can be expressed as follows
\begin{equation}
    S(\vec{G}) \approx b\left(2\pi \frac{k_x}{K_x},n\right)\,b\left(2\pi \frac{k_y}{K_y},n\right)\,b\left(2\pi \frac{k_z}{K_z},n\right) \, \mathcal{F}_{3D}(Q)
\end{equation}
where $\mathcal{F}_{3D}(Q)$ is the 3D fourier transform of the array Q of interpolated charges on the grids with respect to $t_x,\,t_y\,$ and $t_z$, given by
\begin{equation}
    Q(t_x, t_y, t_z) =  \sum_{i=1}^{N} \sum_{n_1, n_2, n_3} q_i M_n(u_{1i} - t_x - n_1 K_1) \times M_n(u_{2i} - t_y - n_2 K_2) \times M_n(u_{3i} - t_z - n_3 K_3)
\end{equation}
With $ B_{3D}(k_x, k_y, k_z)$ defined as $$ B_{3D}(k_x, k_y, k_z) = \left| b\left(2\pi \frac{k_x}{K_x},n\right) \right|^2 \cdot \left| b\left(2\pi \frac{k_y}{K_y},n\right) \right|^2 \cdot \left| b\left(2\pi \frac{k_z}{K_z},n\right) \right|^2$$
Define $\chi(p_1,p_2,p_3)$ as
\begin{flalign}
    \nonumber \chi(p_x,p_y,p_z) &= \frac{2\pi}{L_xL_yL_z}\sum_{\vec{\mathbf{k}}=-\infty}^{\infty}{}^\prime \frac{1}{|\vec{G}|^2}\,{exp}\left(\frac{-1}{4\alpha^2}|\vec{G}|^2\right)\,\times B_{3D}(k_x,k_y,k_z)
    \\&\quad\quad\quad\times \exp\left( i \frac{2\pi k_x p_x}{K_x} + i \frac{2\pi k_y p_y}{K_y} + i \frac{2\pi k_z p_z}{K_z} \right)
    \\&=\mathcal{F}_{3D}(E_{3D}.B_{3D})(p_x,p_y,p_z)
\end{flalign}
where the screening factor array $E_{3D}$ is defined as
\begin{flalign}
    E_{3D}(k_x,k_y,k_z) &=\frac{2\pi}{L_xL_yL_z} \frac{1}{|\vec{G}|^2}\,{exp}\left(\frac{-1}{4\alpha^2}|\vec{G}|^2\right)
\end{flalign}
Note that $E_{3D}\cdot B_{3D} = \mathcal{F}^{-1}_{3D}(\chi)$. The approximate reciprocal space sum is thus given as
\begin{flalign}
    \nonumber(4\pi\epsilon_o)U^{LR}  & \approx \frac{2\pi}{L_xL_yL_z}\sum_{\mathbf{k}=-\infty}^{\infty}{}^{\prime}\frac{1}{|\vec{G}|^2}\,{exp}\left(\frac{-1}{4\alpha^2}|\vec{G}|^2\right)\, B_{3D}(k_x,k_y,k_z)  \\
    &\quad\quad\quad\quad\quad\cdot \left|\mathcal{F}_{3D}(Q)(k_x,k_y,k_z)\right|^2 \\ \label{eq:reci3DSPME}
    \nonumber &= \sum_{k_x=0}^{K_1-1} \sum_{k_y=0}^{K_2-1} \sum_{k_z=0}^{K_3-1}\mathcal{F}^{-1}_{3D}(\chi)(k_x,k_y,k_z)\cdot \mathcal{F}_{3D}(Q)(k_x,k_y,k_z) \\
    \nonumber &\quad\quad\quad\quad\quad\cdot K_xK_yK_z\,\mathcal{F}^{-1}_{3D}(Q)(k_x,k_y,k_z) \\
    &=\sum_{k_x=0}^{K_1-1} \sum_{k_y=0}^{K_2-1} \sum_{k_z=0}^{K_3-1} Q(k_x,k_y,k_z) \cdot (\chi \star Q)(k_x,k_y,k_z)
\end{flalign}
where $\star$ is the convolution operation expressed as
\begin{flalign}
    \chi \star Q(k_x,k_y,k_z) &= \sum_{k_x^\prime=0}^{K_1-1} \sum_{k_y^\prime=0}^{K_2-1} \sum_{k_z^\prime=0}^{K_3-1} \chi (k_x^\prime,k_y^\prime,k_z^\prime) \times Q (k_x-k_x^\prime,k_y-k_y^\prime,k_z-k_z^\prime)
\end{flalign}
As the array \(\chi\) is independent of the spatial orientation of the particles, the reciprocal space force  can therefore be expressed as follows
\begin{flalign}
    \frac{\partial U^{LR}}{\partial r_{\beta i}} &= 2\sum_{k_x=0}^{K_1-1} \sum_{k_y=0}^{K_2-1} \sum_{k_z=0}^{K_3-1} \frac{\partial Q(k_x,k_y,k_z)}{\partial r_{\beta i}} \cdot (\chi \star Q)(k_x,k_y,k_z)
\end{flalign}

\section{Slab-Type (Two-Dimensional Periodic) Systems}
Define $\sigma(k_{x}) = \frac{2\pi k_{x}}{L_{x}}$ and $\psi(k_{y}) = \frac{2\pi k_{y}}{L_{y}}$, using these definitions, the structure factor is represented as
\begin{flalign}
    S(\vec{K},h) =\sum_{j=1}^{n_p}q_j\,exp[i(\sigma(k_x)x_{j} + \psi(k_y)y_{j}+hz_{j})] \label{eq:structurefactor}
\end{flalign}
The Coulombic interaction in Eq.~(\ref{eq:coul}) for a two-dimensional periodic system is expressed using the Ewald summation as
\begin{flalign}
    \nonumber (4\pi\epsilon_o)U &= U^{LR}_{k\neq0} +U^{LR}_{k=0} + U^{S} + U^{SR}
\end{flalign}
Each of these terms is written as
\begin{flalign}
    U^{LR}_{k\neq0}  &= \frac{1}{L_xL_y} \sum_{\mathbf{k}=-\infty}^{\infty} {}^\prime 
    \int_{-\infty}^{\infty} dh\, \frac{\,{exp}\left(-\frac{\sigma^2+\psi^2 +h^2}{4\alpha^2}\right)}
    {\sigma^2+\psi^2 +h^2} \, |S(\vec{K},h)|^2  \\
     U^{LR}_{k=0} &= \frac{2\sqrt{\pi}}{L_xL_y} \sum_{i=1}^{n_p}\sum_{j=1}^{n_p}q_i q_j
    \left[\frac{1-\,{exp}(-z_{ij}^2\alpha^2)}{\alpha}+\sqrt{\pi}z_{ij}\,{erf}(\alpha z_{ij})\right] \\
    U^S&= -\frac{\alpha}{\sqrt{\pi}}\sum_{i=1}^{n_p} q_i^2 \\
    U_{SR}&=\frac{1}{2}\sum_{i=1}^{n_p}{}^\prime\sum_{j=1}^{n_p} q_i q_j\frac{erfc(\alpha|\vec{r_j}-\vec{r_i}|)}{|\vec{r_j}-\vec{r_i}|}
\end{flalign}
The above formulations were presented by Kawata \textit{et al}. For the reciprocal space summation, the absence of periodicity along the $z$-direction leads to an integral over this coordinate. Consequently, the summation over $k_z$, discrete in a fully periodic system, becomes a continuous Fourier transform. The real-space contribution to the Ewald summation remains the same in the three-dimensional and two-dimensional periodic cases, with the only difference being the choice of the minimum image convention appropriate to each geometry. 

\subsection{2D-Particle Mesh Ewald}
In the 2D-PME method, the structure factor in the reciprocal space sum is expressed as
\begin{equation}
\exp(2\pi i \mathbf{k} \cdot \mathbf{r'}) =
\exp\left(2\pi i \frac{k_x u_1}{K_1} \right)
\exp\left(2\pi i \frac{k_y u_2}{K_2} \right),
\end{equation}
where $\mathbf{r'}$ denotes the component of $\mathbf{r}$ confined to the (x, y) plane, and $\mathbf{u} = \mathbf{K}.\mathbf{(r')^*}$ represents the scaled fractional coordinates, with $\mathbf{K}$ being the vector corresponding to the grid points.
These exponential terms are evaluated using exponential B-splines, as defined in Eq. (\ref{eq:bspline}). Meanwhile, the $z$-direction term of the structure factor is approximated as
\begin{flalign}
    exp(ihz) \approx b(h,n)\times \sum_{t = -\infty}^{\infty} M_n[z - t]exp(iht)
\end{flalign}
The 2D structure factor from Eq. (\ref{eq:structurefactor}) can now be approximated as
\begin{flalign}
    \nonumber S(\vec{K},h) & \approx b\left(2\pi \frac{k_x}{K_x},n\right)\,b\left(2\pi \frac{k_y}{K_y},n\right)\, b(h,n) \,
    \\ & \sum_{t_x=0}^{K_x-1} \sum_{t_y=0}^{K_y-1} \sum_{t_z=-\infty}^{\infty}\left[ Q(t_x, t_y, t_z) \times \exp\left( i \frac{2\pi k_x t_x}{K_x} + i \frac{2\pi k_y t_y}{K_y} + i h t_z \right) \right]
\end{flalign}
The array Q of interpolated charges on the grids, given by
\begin{equation}
    Q(t_x, t_y, t_z) = \sum_{i=1}^{N} \sum_{n_1, n_2} q_i M_n(u_{1i} - t_x - n_1 K_1) \times M_n(u_{2i} - t_y - n_2 K_2) \times M_n(z_{i} - t_z)
\end{equation}
With $ B_{2D}(k_x, k_y, h)$ defined as $$ B_{2D}(k_x, k_y, h) = \left| b\left(2\pi \frac{k_x}{K_x},n\right) \right|^2 \cdot \left| b\left(2\pi \frac{k_y}{K_y},n\right) \right|^2 \cdot \left| b\left(h,n\right) \right|^2$$ 
The 2D-PME long-range interaction can be expressed as
\begin{flalign}
    (4\pi\epsilon_o)U^{LR} &\approx \frac{1}{L_xL_y} \sum_{k_x,k_y = -\infty}^{\infty}\int_{-\infty}^{\infty}dh\, B_{2D}(k_x,k_y,h)\frac{\,{exp}\left(-\frac{\sigma^2+\psi^2 +h^2}{4\alpha^2}\right)}{\sigma^2+\psi^2 +h^2} \, \left|\tilde{Q}(k_x,k_y,h)\right|^2
\end{flalign}
where $\tilde{Q}(k_x,k_y,h)$ is the 1D fourier integral of ${Q^\prime}(k_x,k_y,t_z)$ with respect to $t_z$ and ${Q^\prime}(k_x,k_y,t_z)$ is the 2D discrete fourier transform of $Q(t_x, t_y, t_z)$ with respect to $t_x$ and $t_y$. Define array $\theta(p_1,p_2,p_3)$ as
\begin{flalign}
    \nonumber \theta(p_x,p_y,p_z) &= \frac{1}{L_xL_y}\sum_{k_x,k_y = -\infty}^{\infty}{}^\prime \int_{-\infty}^{\infty}dh\, \frac{\,{exp}\left(-\frac{\sigma^2+\psi^2 +h^2}{4\alpha^2}\right)}{\sigma^2+\psi^2 +h^2} \,B_{2D}(k_x,k_y,h)
    \\&\quad\quad\quad\times \exp\left( i \frac{2\pi k_x p_x}{K_x} + i \frac{2\pi k_y p_y}{K_y} +  i h p_z \right)
    % \\&=\mathcal{F}_{3D}(E_{3D}.B_{3D})(p_x,p_y,p_z)
\end{flalign}
gives a simplified expression of the reciprocal space energy as
\begin{flalign}
     (4\pi\epsilon_o)U^{LR} &=\sum_{k_x=0}^{K_1-1} \sum_{k_y=0}^{K_2-1} \sum_{k_z=-\infty}^{\infty} Q(k_x,k_y,k_z) \cdot (\theta \star Q)(k_x,k_y,k_z)
\end{flalign}
where $\star$ is the convolution operation expressed as
\begin{flalign}
    \theta \star Q(k_x,k_y,k_z) &= \sum_{k_x^\prime=0}^{K_1-1} \sum_{k_y^\prime=0}^{K_2-1} \sum_{k_z^\prime=-\infty}^{\infty} \theta (k_x^\prime,k_y^\prime,k_z^\prime) \times Q (k_x-k_x^\prime,k_y-k_y^\prime,k_z-k_z^\prime)
\end{flalign}
The 2D-reciprocal space force can therefore be derived as
\begin{flalign}
      \frac{\partial U^{LR}}{\partial r_{\beta i}} &= 2\sum_{k_x=0}^{K_1-1} \sum_{k_y=0}^{K_2-1} \sum_{k_z^\prime=-\infty}^{\infty} \frac{\partial Q(k_x,k_y,k_z)}{\partial r_{\beta i}} \cdot (\theta \star Q)(k_x,k_y,k_z)
\end{flalign}